
% Default to the notebook output style

    


% Inherit from the specified cell style.




    
\documentclass[11pt]{article}

    
    
    \usepackage[T1]{fontenc}
    % Nicer default font (+ math font) than Computer Modern for most use cases
    \usepackage{mathpazo}

    % Basic figure setup, for now with no caption control since it's done
    % automatically by Pandoc (which extracts ![](path) syntax from Markdown).
    \usepackage{graphicx}
    % We will generate all images so they have a width \maxwidth. This means
    % that they will get their normal width if they fit onto the page, but
    % are scaled down if they would overflow the margins.
    \makeatletter
    \def\maxwidth{\ifdim\Gin@nat@width>\linewidth\linewidth
    \else\Gin@nat@width\fi}
    \makeatother
    \let\Oldincludegraphics\includegraphics
    % Set max figure width to be 80% of text width, for now hardcoded.
    \renewcommand{\includegraphics}[1]{\Oldincludegraphics[width=.8\maxwidth]{#1}}
    % Ensure that by default, figures have no caption (until we provide a
    % proper Figure object with a Caption API and a way to capture that
    % in the conversion process - todo).
    \usepackage{caption}
    \DeclareCaptionLabelFormat{nolabel}{}
    \captionsetup{labelformat=nolabel}

    \usepackage{adjustbox} % Used to constrain images to a maximum size 
    \usepackage{xcolor} % Allow colors to be defined
    \usepackage{enumerate} % Needed for markdown enumerations to work
    \usepackage{geometry} % Used to adjust the document margins
    \usepackage{amsmath} % Equations
    \usepackage{amssymb} % Equations
    \usepackage{textcomp} % defines textquotesingle
    % Hack from http://tex.stackexchange.com/a/47451/13684:
    \AtBeginDocument{%
        \def\PYZsq{\textquotesingle}% Upright quotes in Pygmentized code
    }
    \usepackage{upquote} % Upright quotes for verbatim code
    \usepackage{eurosym} % defines \euro
    \usepackage[mathletters]{ucs} % Extended unicode (utf-8) support
    \usepackage[utf8x]{inputenc} % Allow utf-8 characters in the tex document
    \usepackage{fancyvrb} % verbatim replacement that allows latex
    \usepackage{grffile} % extends the file name processing of package graphics 
                         % to support a larger range 
    % The hyperref package gives us a pdf with properly built
    % internal navigation ('pdf bookmarks' for the table of contents,
    % internal cross-reference links, web links for URLs, etc.)
    \usepackage{hyperref}
    \usepackage{longtable} % longtable support required by pandoc >1.10
    \usepackage{booktabs}  % table support for pandoc > 1.12.2
    \usepackage[inline]{enumitem} % IRkernel/repr support (it uses the enumerate* environment)
    \usepackage[normalem]{ulem} % ulem is needed to support strikethroughs (\sout)
                                % normalem makes italics be italics, not underlines
    

    
    
    % Colors for the hyperref package
    \definecolor{urlcolor}{rgb}{0,.145,.698}
    \definecolor{linkcolor}{rgb}{.71,0.21,0.01}
    \definecolor{citecolor}{rgb}{.12,.54,.11}

    % ANSI colors
    \definecolor{ansi-black}{HTML}{3E424D}
    \definecolor{ansi-black-intense}{HTML}{282C36}
    \definecolor{ansi-red}{HTML}{E75C58}
    \definecolor{ansi-red-intense}{HTML}{B22B31}
    \definecolor{ansi-green}{HTML}{00A250}
    \definecolor{ansi-green-intense}{HTML}{007427}
    \definecolor{ansi-yellow}{HTML}{DDB62B}
    \definecolor{ansi-yellow-intense}{HTML}{B27D12}
    \definecolor{ansi-blue}{HTML}{208FFB}
    \definecolor{ansi-blue-intense}{HTML}{0065CA}
    \definecolor{ansi-magenta}{HTML}{D160C4}
    \definecolor{ansi-magenta-intense}{HTML}{A03196}
    \definecolor{ansi-cyan}{HTML}{60C6C8}
    \definecolor{ansi-cyan-intense}{HTML}{258F8F}
    \definecolor{ansi-white}{HTML}{C5C1B4}
    \definecolor{ansi-white-intense}{HTML}{A1A6B2}

    % commands and environments needed by pandoc snippets
    % extracted from the output of `pandoc -s`
    \providecommand{\tightlist}{%
      \setlength{\itemsep}{0pt}\setlength{\parskip}{0pt}}
    \DefineVerbatimEnvironment{Highlighting}{Verbatim}{commandchars=\\\{\}}
    % Add ',fontsize=\small' for more characters per line
    \newenvironment{Shaded}{}{}
    \newcommand{\KeywordTok}[1]{\textcolor[rgb]{0.00,0.44,0.13}{\textbf{{#1}}}}
    \newcommand{\DataTypeTok}[1]{\textcolor[rgb]{0.56,0.13,0.00}{{#1}}}
    \newcommand{\DecValTok}[1]{\textcolor[rgb]{0.25,0.63,0.44}{{#1}}}
    \newcommand{\BaseNTok}[1]{\textcolor[rgb]{0.25,0.63,0.44}{{#1}}}
    \newcommand{\FloatTok}[1]{\textcolor[rgb]{0.25,0.63,0.44}{{#1}}}
    \newcommand{\CharTok}[1]{\textcolor[rgb]{0.25,0.44,0.63}{{#1}}}
    \newcommand{\StringTok}[1]{\textcolor[rgb]{0.25,0.44,0.63}{{#1}}}
    \newcommand{\CommentTok}[1]{\textcolor[rgb]{0.38,0.63,0.69}{\textit{{#1}}}}
    \newcommand{\OtherTok}[1]{\textcolor[rgb]{0.00,0.44,0.13}{{#1}}}
    \newcommand{\AlertTok}[1]{\textcolor[rgb]{1.00,0.00,0.00}{\textbf{{#1}}}}
    \newcommand{\FunctionTok}[1]{\textcolor[rgb]{0.02,0.16,0.49}{{#1}}}
    \newcommand{\RegionMarkerTok}[1]{{#1}}
    \newcommand{\ErrorTok}[1]{\textcolor[rgb]{1.00,0.00,0.00}{\textbf{{#1}}}}
    \newcommand{\NormalTok}[1]{{#1}}
    
    % Additional commands for more recent versions of Pandoc
    \newcommand{\ConstantTok}[1]{\textcolor[rgb]{0.53,0.00,0.00}{{#1}}}
    \newcommand{\SpecialCharTok}[1]{\textcolor[rgb]{0.25,0.44,0.63}{{#1}}}
    \newcommand{\VerbatimStringTok}[1]{\textcolor[rgb]{0.25,0.44,0.63}{{#1}}}
    \newcommand{\SpecialStringTok}[1]{\textcolor[rgb]{0.73,0.40,0.53}{{#1}}}
    \newcommand{\ImportTok}[1]{{#1}}
    \newcommand{\DocumentationTok}[1]{\textcolor[rgb]{0.73,0.13,0.13}{\textit{{#1}}}}
    \newcommand{\AnnotationTok}[1]{\textcolor[rgb]{0.38,0.63,0.69}{\textbf{\textit{{#1}}}}}
    \newcommand{\CommentVarTok}[1]{\textcolor[rgb]{0.38,0.63,0.69}{\textbf{\textit{{#1}}}}}
    \newcommand{\VariableTok}[1]{\textcolor[rgb]{0.10,0.09,0.49}{{#1}}}
    \newcommand{\ControlFlowTok}[1]{\textcolor[rgb]{0.00,0.44,0.13}{\textbf{{#1}}}}
    \newcommand{\OperatorTok}[1]{\textcolor[rgb]{0.40,0.40,0.40}{{#1}}}
    \newcommand{\BuiltInTok}[1]{{#1}}
    \newcommand{\ExtensionTok}[1]{{#1}}
    \newcommand{\PreprocessorTok}[1]{\textcolor[rgb]{0.74,0.48,0.00}{{#1}}}
    \newcommand{\AttributeTok}[1]{\textcolor[rgb]{0.49,0.56,0.16}{{#1}}}
    \newcommand{\InformationTok}[1]{\textcolor[rgb]{0.38,0.63,0.69}{\textbf{\textit{{#1}}}}}
    \newcommand{\WarningTok}[1]{\textcolor[rgb]{0.38,0.63,0.69}{\textbf{\textit{{#1}}}}}
    
    
    % Define a nice break command that doesn't care if a line doesn't already
    % exist.
    \def\br{\hspace*{\fill} \\* }
    % Math Jax compatability definitions
    \def\gt{>}
    \def\lt{<}
    % Document parameters
    \title{Homework 5}
    
    
    

    % Pygments definitions
    
\makeatletter
\def\PY@reset{\let\PY@it=\relax \let\PY@bf=\relax%
    \let\PY@ul=\relax \let\PY@tc=\relax%
    \let\PY@bc=\relax \let\PY@ff=\relax}
\def\PY@tok#1{\csname PY@tok@#1\endcsname}
\def\PY@toks#1+{\ifx\relax#1\empty\else%
    \PY@tok{#1}\expandafter\PY@toks\fi}
\def\PY@do#1{\PY@bc{\PY@tc{\PY@ul{%
    \PY@it{\PY@bf{\PY@ff{#1}}}}}}}
\def\PY#1#2{\PY@reset\PY@toks#1+\relax+\PY@do{#2}}

\expandafter\def\csname PY@tok@w\endcsname{\def\PY@tc##1{\textcolor[rgb]{0.73,0.73,0.73}{##1}}}
\expandafter\def\csname PY@tok@c\endcsname{\let\PY@it=\textit\def\PY@tc##1{\textcolor[rgb]{0.25,0.50,0.50}{##1}}}
\expandafter\def\csname PY@tok@cp\endcsname{\def\PY@tc##1{\textcolor[rgb]{0.74,0.48,0.00}{##1}}}
\expandafter\def\csname PY@tok@k\endcsname{\let\PY@bf=\textbf\def\PY@tc##1{\textcolor[rgb]{0.00,0.50,0.00}{##1}}}
\expandafter\def\csname PY@tok@kp\endcsname{\def\PY@tc##1{\textcolor[rgb]{0.00,0.50,0.00}{##1}}}
\expandafter\def\csname PY@tok@kt\endcsname{\def\PY@tc##1{\textcolor[rgb]{0.69,0.00,0.25}{##1}}}
\expandafter\def\csname PY@tok@o\endcsname{\def\PY@tc##1{\textcolor[rgb]{0.40,0.40,0.40}{##1}}}
\expandafter\def\csname PY@tok@ow\endcsname{\let\PY@bf=\textbf\def\PY@tc##1{\textcolor[rgb]{0.67,0.13,1.00}{##1}}}
\expandafter\def\csname PY@tok@nb\endcsname{\def\PY@tc##1{\textcolor[rgb]{0.00,0.50,0.00}{##1}}}
\expandafter\def\csname PY@tok@nf\endcsname{\def\PY@tc##1{\textcolor[rgb]{0.00,0.00,1.00}{##1}}}
\expandafter\def\csname PY@tok@nc\endcsname{\let\PY@bf=\textbf\def\PY@tc##1{\textcolor[rgb]{0.00,0.00,1.00}{##1}}}
\expandafter\def\csname PY@tok@nn\endcsname{\let\PY@bf=\textbf\def\PY@tc##1{\textcolor[rgb]{0.00,0.00,1.00}{##1}}}
\expandafter\def\csname PY@tok@ne\endcsname{\let\PY@bf=\textbf\def\PY@tc##1{\textcolor[rgb]{0.82,0.25,0.23}{##1}}}
\expandafter\def\csname PY@tok@nv\endcsname{\def\PY@tc##1{\textcolor[rgb]{0.10,0.09,0.49}{##1}}}
\expandafter\def\csname PY@tok@no\endcsname{\def\PY@tc##1{\textcolor[rgb]{0.53,0.00,0.00}{##1}}}
\expandafter\def\csname PY@tok@nl\endcsname{\def\PY@tc##1{\textcolor[rgb]{0.63,0.63,0.00}{##1}}}
\expandafter\def\csname PY@tok@ni\endcsname{\let\PY@bf=\textbf\def\PY@tc##1{\textcolor[rgb]{0.60,0.60,0.60}{##1}}}
\expandafter\def\csname PY@tok@na\endcsname{\def\PY@tc##1{\textcolor[rgb]{0.49,0.56,0.16}{##1}}}
\expandafter\def\csname PY@tok@nt\endcsname{\let\PY@bf=\textbf\def\PY@tc##1{\textcolor[rgb]{0.00,0.50,0.00}{##1}}}
\expandafter\def\csname PY@tok@nd\endcsname{\def\PY@tc##1{\textcolor[rgb]{0.67,0.13,1.00}{##1}}}
\expandafter\def\csname PY@tok@s\endcsname{\def\PY@tc##1{\textcolor[rgb]{0.73,0.13,0.13}{##1}}}
\expandafter\def\csname PY@tok@sd\endcsname{\let\PY@it=\textit\def\PY@tc##1{\textcolor[rgb]{0.73,0.13,0.13}{##1}}}
\expandafter\def\csname PY@tok@si\endcsname{\let\PY@bf=\textbf\def\PY@tc##1{\textcolor[rgb]{0.73,0.40,0.53}{##1}}}
\expandafter\def\csname PY@tok@se\endcsname{\let\PY@bf=\textbf\def\PY@tc##1{\textcolor[rgb]{0.73,0.40,0.13}{##1}}}
\expandafter\def\csname PY@tok@sr\endcsname{\def\PY@tc##1{\textcolor[rgb]{0.73,0.40,0.53}{##1}}}
\expandafter\def\csname PY@tok@ss\endcsname{\def\PY@tc##1{\textcolor[rgb]{0.10,0.09,0.49}{##1}}}
\expandafter\def\csname PY@tok@sx\endcsname{\def\PY@tc##1{\textcolor[rgb]{0.00,0.50,0.00}{##1}}}
\expandafter\def\csname PY@tok@m\endcsname{\def\PY@tc##1{\textcolor[rgb]{0.40,0.40,0.40}{##1}}}
\expandafter\def\csname PY@tok@gh\endcsname{\let\PY@bf=\textbf\def\PY@tc##1{\textcolor[rgb]{0.00,0.00,0.50}{##1}}}
\expandafter\def\csname PY@tok@gu\endcsname{\let\PY@bf=\textbf\def\PY@tc##1{\textcolor[rgb]{0.50,0.00,0.50}{##1}}}
\expandafter\def\csname PY@tok@gd\endcsname{\def\PY@tc##1{\textcolor[rgb]{0.63,0.00,0.00}{##1}}}
\expandafter\def\csname PY@tok@gi\endcsname{\def\PY@tc##1{\textcolor[rgb]{0.00,0.63,0.00}{##1}}}
\expandafter\def\csname PY@tok@gr\endcsname{\def\PY@tc##1{\textcolor[rgb]{1.00,0.00,0.00}{##1}}}
\expandafter\def\csname PY@tok@ge\endcsname{\let\PY@it=\textit}
\expandafter\def\csname PY@tok@gs\endcsname{\let\PY@bf=\textbf}
\expandafter\def\csname PY@tok@gp\endcsname{\let\PY@bf=\textbf\def\PY@tc##1{\textcolor[rgb]{0.00,0.00,0.50}{##1}}}
\expandafter\def\csname PY@tok@go\endcsname{\def\PY@tc##1{\textcolor[rgb]{0.53,0.53,0.53}{##1}}}
\expandafter\def\csname PY@tok@gt\endcsname{\def\PY@tc##1{\textcolor[rgb]{0.00,0.27,0.87}{##1}}}
\expandafter\def\csname PY@tok@err\endcsname{\def\PY@bc##1{\setlength{\fboxsep}{0pt}\fcolorbox[rgb]{1.00,0.00,0.00}{1,1,1}{\strut ##1}}}
\expandafter\def\csname PY@tok@kc\endcsname{\let\PY@bf=\textbf\def\PY@tc##1{\textcolor[rgb]{0.00,0.50,0.00}{##1}}}
\expandafter\def\csname PY@tok@kd\endcsname{\let\PY@bf=\textbf\def\PY@tc##1{\textcolor[rgb]{0.00,0.50,0.00}{##1}}}
\expandafter\def\csname PY@tok@kn\endcsname{\let\PY@bf=\textbf\def\PY@tc##1{\textcolor[rgb]{0.00,0.50,0.00}{##1}}}
\expandafter\def\csname PY@tok@kr\endcsname{\let\PY@bf=\textbf\def\PY@tc##1{\textcolor[rgb]{0.00,0.50,0.00}{##1}}}
\expandafter\def\csname PY@tok@bp\endcsname{\def\PY@tc##1{\textcolor[rgb]{0.00,0.50,0.00}{##1}}}
\expandafter\def\csname PY@tok@fm\endcsname{\def\PY@tc##1{\textcolor[rgb]{0.00,0.00,1.00}{##1}}}
\expandafter\def\csname PY@tok@vc\endcsname{\def\PY@tc##1{\textcolor[rgb]{0.10,0.09,0.49}{##1}}}
\expandafter\def\csname PY@tok@vg\endcsname{\def\PY@tc##1{\textcolor[rgb]{0.10,0.09,0.49}{##1}}}
\expandafter\def\csname PY@tok@vi\endcsname{\def\PY@tc##1{\textcolor[rgb]{0.10,0.09,0.49}{##1}}}
\expandafter\def\csname PY@tok@vm\endcsname{\def\PY@tc##1{\textcolor[rgb]{0.10,0.09,0.49}{##1}}}
\expandafter\def\csname PY@tok@sa\endcsname{\def\PY@tc##1{\textcolor[rgb]{0.73,0.13,0.13}{##1}}}
\expandafter\def\csname PY@tok@sb\endcsname{\def\PY@tc##1{\textcolor[rgb]{0.73,0.13,0.13}{##1}}}
\expandafter\def\csname PY@tok@sc\endcsname{\def\PY@tc##1{\textcolor[rgb]{0.73,0.13,0.13}{##1}}}
\expandafter\def\csname PY@tok@dl\endcsname{\def\PY@tc##1{\textcolor[rgb]{0.73,0.13,0.13}{##1}}}
\expandafter\def\csname PY@tok@s2\endcsname{\def\PY@tc##1{\textcolor[rgb]{0.73,0.13,0.13}{##1}}}
\expandafter\def\csname PY@tok@sh\endcsname{\def\PY@tc##1{\textcolor[rgb]{0.73,0.13,0.13}{##1}}}
\expandafter\def\csname PY@tok@s1\endcsname{\def\PY@tc##1{\textcolor[rgb]{0.73,0.13,0.13}{##1}}}
\expandafter\def\csname PY@tok@mb\endcsname{\def\PY@tc##1{\textcolor[rgb]{0.40,0.40,0.40}{##1}}}
\expandafter\def\csname PY@tok@mf\endcsname{\def\PY@tc##1{\textcolor[rgb]{0.40,0.40,0.40}{##1}}}
\expandafter\def\csname PY@tok@mh\endcsname{\def\PY@tc##1{\textcolor[rgb]{0.40,0.40,0.40}{##1}}}
\expandafter\def\csname PY@tok@mi\endcsname{\def\PY@tc##1{\textcolor[rgb]{0.40,0.40,0.40}{##1}}}
\expandafter\def\csname PY@tok@il\endcsname{\def\PY@tc##1{\textcolor[rgb]{0.40,0.40,0.40}{##1}}}
\expandafter\def\csname PY@tok@mo\endcsname{\def\PY@tc##1{\textcolor[rgb]{0.40,0.40,0.40}{##1}}}
\expandafter\def\csname PY@tok@ch\endcsname{\let\PY@it=\textit\def\PY@tc##1{\textcolor[rgb]{0.25,0.50,0.50}{##1}}}
\expandafter\def\csname PY@tok@cm\endcsname{\let\PY@it=\textit\def\PY@tc##1{\textcolor[rgb]{0.25,0.50,0.50}{##1}}}
\expandafter\def\csname PY@tok@cpf\endcsname{\let\PY@it=\textit\def\PY@tc##1{\textcolor[rgb]{0.25,0.50,0.50}{##1}}}
\expandafter\def\csname PY@tok@c1\endcsname{\let\PY@it=\textit\def\PY@tc##1{\textcolor[rgb]{0.25,0.50,0.50}{##1}}}
\expandafter\def\csname PY@tok@cs\endcsname{\let\PY@it=\textit\def\PY@tc##1{\textcolor[rgb]{0.25,0.50,0.50}{##1}}}

\def\PYZbs{\char`\\}
\def\PYZus{\char`\_}
\def\PYZob{\char`\{}
\def\PYZcb{\char`\}}
\def\PYZca{\char`\^}
\def\PYZam{\char`\&}
\def\PYZlt{\char`\<}
\def\PYZgt{\char`\>}
\def\PYZsh{\char`\#}
\def\PYZpc{\char`\%}
\def\PYZdl{\char`\$}
\def\PYZhy{\char`\-}
\def\PYZsq{\char`\'}
\def\PYZdq{\char`\"}
\def\PYZti{\char`\~}
% for compatibility with earlier versions
\def\PYZat{@}
\def\PYZlb{[}
\def\PYZrb{]}
\makeatother


    % Exact colors from NB
    \definecolor{incolor}{rgb}{0.0, 0.0, 0.5}
    \definecolor{outcolor}{rgb}{0.545, 0.0, 0.0}



    
    % Prevent overflowing lines due to hard-to-break entities
    \sloppy 
    % Setup hyperref package
    \hypersetup{
      breaklinks=true,  % so long urls are correctly broken across lines
      colorlinks=true,
      urlcolor=urlcolor,
      linkcolor=linkcolor,
      citecolor=citecolor,
      }
    % Slightly bigger margins than the latex defaults
    
    \geometry{verbose,tmargin=1in,bmargin=1in,lmargin=1in,rmargin=1in}
    
    

    \begin{document}
    
    
    \maketitle
    
    

    
    \hypertarget{homework-5---independent-component-analysis}{%
\section{Homework 5 - Independent Component
Analysis}\label{homework-5---independent-component-analysis}}

    Brennan McFarland\\
bfm21

    In this assignment I implement gradient-based Independent Component
Analysis for Blind Source Separation and apply it to the cocktail party
problem. My algorithm, implemented in the bsslib.py file, applies the
BSS algorithm to a matrix of unseparated data from mixed sources and
performs gradient descent on the mixing matrix until convergence, which
it then uses to recover the original source data. I first demonstrate
this algorithm on a randomly generated Laplacian matrix, and then use it
to separate audio files from mixed soundtracks.

    \hypertarget{loading-the-data}{%
\subsubsection{Loading the Data}\label{loading-the-data}}

    I will apply the BSS algorithm to a mixture of the following speech and
classical music sound files:

    \begin{Verbatim}[commandchars=\\\{\}]
{\color{incolor}In [{\color{incolor}1}]:} \PY{k+kn}{from} \PY{n+nn}{scipy}\PY{n+nn}{.}\PY{n+nn}{io} \PY{k}{import} \PY{n}{wavfile}
        
        \PY{n}{srate}\PY{p}{,} \PY{n}{dataBach} \PY{o}{=} \PY{n}{wavfile}\PY{o}{.}\PY{n}{read}\PY{p}{(}\PY{l+s+s1}{\PYZsq{}}\PY{l+s+s1}{data/bach.wav}\PY{l+s+s1}{\PYZsq{}}\PY{p}{)}
        \PY{n}{\PYZus{}}\PY{p}{,} \PY{n}{dataSpeech} \PY{o}{=} \PY{n}{wavfile}\PY{o}{.}\PY{n}{read}\PY{p}{(}\PY{l+s+s1}{\PYZsq{}}\PY{l+s+s1}{data/speech.wav}\PY{l+s+s1}{\PYZsq{}}\PY{p}{)}
\end{Verbatim}


    \hypertarget{normalizing-the-data}{%
\subsubsection{Normalizing the Data}\label{normalizing-the-data}}

    This function converts the sound data to a form we can more easily
manipulate by normalizing the soundtracks as numpy arrays:

    \begin{Verbatim}[commandchars=\\\{\}]
{\color{incolor}In [{\color{incolor}2}]:} \PY{k+kn}{import} \PY{n+nn}{numpy} \PY{k}{as} \PY{n+nn}{np}
        
        \PY{k}{def} \PY{n+nf}{audionorm}\PY{p}{(}\PY{n}{data}\PY{p}{)}\PY{p}{:}
            \PY{c+c1}{\PYZsh{} ensure data is ndarray with float numbers}
            \PY{n}{data} \PY{o}{=} \PY{n}{np}\PY{o}{.}\PY{n}{asarray}\PY{p}{(}\PY{n}{data}\PY{p}{)}\PY{o}{.}\PY{n}{astype}\PY{p}{(}\PY{l+s+s1}{\PYZsq{}}\PY{l+s+s1}{float}\PY{l+s+s1}{\PYZsq{}}\PY{p}{)}
            \PY{c+c1}{\PYZsh{} calculate lower and upper bound}
            \PY{n}{lbound}\PY{p}{,} \PY{n}{ubound} \PY{o}{=} \PY{n}{np}\PY{o}{.}\PY{n}{min}\PY{p}{(}\PY{n}{data}\PY{p}{)}\PY{p}{,} \PY{n}{np}\PY{o}{.}\PY{n}{max}\PY{p}{(}\PY{n}{data}\PY{p}{)}
            \PY{k}{if} \PY{n}{lbound} \PY{o}{==} \PY{n}{ubound}\PY{p}{:}
                \PY{n}{offset} \PY{o}{=} \PY{n}{lbound}
                \PY{n}{scalar} \PY{o}{=} \PY{l+m+mi}{1}
                \PY{n}{data} \PY{o}{=} \PY{n}{np}\PY{o}{.}\PY{n}{zeros}\PY{p}{(}\PY{n}{size}\PY{o}{=}\PY{n}{data}\PY{o}{.}\PY{n}{shape}\PY{p}{)}
            \PY{k}{else}\PY{p}{:}
                \PY{n}{offset} \PY{o}{=} \PY{p}{(}\PY{n}{lbound} \PY{o}{+} \PY{n}{ubound}\PY{p}{)} \PY{o}{/} \PY{l+m+mi}{2}
                \PY{n}{scalar} \PY{o}{=} \PY{l+m+mi}{1} \PY{o}{/} \PY{p}{(}\PY{n}{ubound} \PY{o}{\PYZhy{}} \PY{n}{lbound}\PY{p}{)}
                \PY{n}{data} \PY{o}{=} \PY{p}{(}\PY{n}{data} \PY{o}{\PYZhy{}} \PY{n}{offset}\PY{p}{)} \PY{o}{*} \PY{n}{scalar}
            \PY{c+c1}{\PYZsh{} return normalized data}
            \PY{k}{return} \PY{n}{data}
        
        \PY{n}{gtruthS} \PY{o}{=} \PY{n}{audionorm}\PY{p}{(}\PY{p}{[}\PY{n}{dataBach}\PY{p}{,} \PY{n}{dataSpeech}\PY{p}{]}\PY{p}{)}
\end{Verbatim}


    \hypertarget{mixing-audio}{%
\subsubsection{Mixing Audio}\label{mixing-audio}}

    Here we apply a simple mixing function, mixing the audio data via
multiplying the source soundtracks (in a 2xn matrix, 2 for the number of
soundtracks and n for the number of data points in the audio clip) by a
random mixing matrix transformation. This produces another 2xn matrix
with two new soundtracks representing the mixed sounds. We then display
the data output along with our coordinate axes for the mixing matrix and
save the mixed audio to their respective files.

    \begin{Verbatim}[commandchars=\\\{\}]
{\color{incolor}In [{\color{incolor}3}]:} \PY{k}{def} \PY{n+nf}{simpleMixer}\PY{p}{(}\PY{n}{S}\PY{p}{)}\PY{p}{:}
            \PY{n}{nchannel} \PY{o}{=} \PY{n}{S}\PY{o}{.}\PY{n}{shape}\PY{p}{[}\PY{l+m+mi}{0}\PY{p}{]}
            \PY{c+c1}{\PYZsh{} generate a random matrix}
            \PY{n}{A} \PY{o}{=} \PY{n}{np}\PY{o}{.}\PY{n}{random}\PY{o}{.}\PY{n}{uniform}\PY{p}{(}\PY{n}{size} \PY{o}{=} \PY{p}{(}\PY{n}{nchannel}\PY{p}{,}\PY{n}{nchannel}\PY{p}{)}\PY{p}{)}
            \PY{c+c1}{\PYZsh{} generate mixed audio data}
            \PY{n}{X} \PY{o}{=} \PY{n}{A}\PY{o}{.}\PY{n}{dot}\PY{p}{(}\PY{n}{S}\PY{p}{)}
            
            \PY{k}{return} \PY{n}{X}\PY{p}{,} \PY{n}{A}
\end{Verbatim}


    \begin{Verbatim}[commandchars=\\\{\}]
{\color{incolor}In [{\color{incolor}4}]:} \PY{o}{\PYZpc{}}\PY{k}{matplotlib} inline
        
        \PY{k+kn}{import} \PY{n+nn}{matplotlib}\PY{n+nn}{.}\PY{n+nn}{pyplot} \PY{k}{as} \PY{n+nn}{plt}
        
        \PY{k}{def} \PY{n+nf}{drawDataWithMixingMatrix}\PY{p}{(}\PY{n}{data}\PY{p}{,} \PY{n}{mat}\PY{p}{)}\PY{p}{:}
            \PY{c+c1}{\PYZsh{} plot data points}
            \PY{n}{plt}\PY{o}{.}\PY{n}{scatter}\PY{p}{(}\PY{n}{data}\PY{p}{[}\PY{l+m+mi}{0}\PY{p}{]}\PY{p}{,} \PY{n}{data}\PY{p}{[}\PY{l+m+mi}{1}\PY{p}{]}\PY{p}{,} \PY{n}{s}\PY{o}{=}\PY{l+m+mi}{1}\PY{p}{)}
            \PY{c+c1}{\PYZsh{} calculate axis length}
            \PY{n}{lenAxis} \PY{o}{=} \PY{n}{np}\PY{o}{.}\PY{n}{sqrt}\PY{p}{(}\PY{n}{np}\PY{o}{.}\PY{n}{sum}\PY{p}{(}\PY{n}{np}\PY{o}{.}\PY{n}{square}\PY{p}{(}\PY{n}{mat}\PY{p}{)}\PY{p}{,} \PY{n}{axis}\PY{o}{=}\PY{l+m+mi}{0}\PY{p}{)}\PY{p}{)}
            \PY{c+c1}{\PYZsh{} calculate scale for illustration}
            \PY{n}{scale} \PY{o}{=} \PY{n}{np}\PY{o}{.}\PY{n}{min}\PY{p}{(}\PY{n}{np}\PY{o}{.}\PY{n}{max}\PY{p}{(}\PY{n}{np}\PY{o}{.}\PY{n}{abs}\PY{p}{(}\PY{n}{data}\PY{p}{)}\PY{p}{,} \PY{n}{axis}\PY{o}{=}\PY{l+m+mi}{1}\PY{p}{)} \PY{o}{/} \PY{n}{lenAxis}\PY{o}{.}\PY{n}{T}\PY{p}{)}
            \PY{c+c1}{\PYZsh{} draw axis as arrow}
            \PY{n}{plt}\PY{o}{.}\PY{n}{arrow}\PY{p}{(}\PY{l+m+mi}{0}\PY{p}{,} \PY{l+m+mi}{0}\PY{p}{,} \PY{n}{scale} \PY{o}{*} \PY{n}{mat}\PY{p}{[}\PY{l+m+mi}{0}\PY{p}{,}\PY{l+m+mi}{0}\PY{p}{]}\PY{p}{,} \PY{n}{scale} \PY{o}{*} \PY{n}{mat}\PY{p}{[}\PY{l+m+mi}{1}\PY{p}{,}\PY{l+m+mi}{0}\PY{p}{]}\PY{p}{,} \PY{n}{shape}\PY{o}{=}\PY{l+s+s1}{\PYZsq{}}\PY{l+s+s1}{full}\PY{l+s+s1}{\PYZsq{}}\PY{p}{,} \PY{n}{color}\PY{o}{=}\PY{l+s+s1}{\PYZsq{}}\PY{l+s+s1}{r}\PY{l+s+s1}{\PYZsq{}}\PY{p}{)}
            \PY{n}{plt}\PY{o}{.}\PY{n}{arrow}\PY{p}{(}\PY{l+m+mi}{0}\PY{p}{,} \PY{l+m+mi}{0}\PY{p}{,} \PY{n}{scale} \PY{o}{*} \PY{n}{mat}\PY{p}{[}\PY{l+m+mi}{0}\PY{p}{,}\PY{l+m+mi}{1}\PY{p}{]}\PY{p}{,} \PY{n}{scale} \PY{o}{*} \PY{n}{mat}\PY{p}{[}\PY{l+m+mi}{1}\PY{p}{,}\PY{l+m+mi}{1}\PY{p}{]}\PY{p}{,} \PY{n}{shape}\PY{o}{=}\PY{l+s+s1}{\PYZsq{}}\PY{l+s+s1}{full}\PY{l+s+s1}{\PYZsq{}}\PY{p}{,} \PY{n}{color}\PY{o}{=}\PY{l+s+s1}{\PYZsq{}}\PY{l+s+s1}{r}\PY{l+s+s1}{\PYZsq{}}\PY{p}{)}
\end{Verbatim}


    \begin{Verbatim}[commandchars=\\\{\}]
{\color{incolor}In [{\color{incolor}5}]:} \PY{n}{X}\PY{p}{,} \PY{n}{gtruthA} \PY{o}{=} \PY{n}{simpleMixer}\PY{p}{(}\PY{n}{gtruthS}\PY{p}{)}
        \PY{n}{drawDataWithMixingMatrix}\PY{p}{(}\PY{n}{X}\PY{p}{,} \PY{n}{gtruthA}\PY{p}{)}
\end{Verbatim}


    \begin{center}
    \adjustimage{max size={0.9\linewidth}{0.9\paperheight}}{output_13_0.png}
    \end{center}
    { \hspace*{\fill} \\}
    
    \begin{Verbatim}[commandchars=\\\{\}]
{\color{incolor}In [{\color{incolor}6}]:} \PY{n}{wavfile}\PY{o}{.}\PY{n}{write}\PY{p}{(}\PY{l+s+s1}{\PYZsq{}}\PY{l+s+s1}{data/mixedTrackA.wav}\PY{l+s+s1}{\PYZsq{}}\PY{p}{,} \PY{n}{srate}\PY{p}{,} \PY{n}{X}\PY{p}{[}\PY{l+m+mi}{0}\PY{p}{]}\PY{p}{)}
        \PY{n}{wavfile}\PY{o}{.}\PY{n}{write}\PY{p}{(}\PY{l+s+s1}{\PYZsq{}}\PY{l+s+s1}{data/mixedTrackB.wav}\PY{l+s+s1}{\PYZsq{}}\PY{p}{,} \PY{n}{srate}\PY{p}{,} \PY{n}{X}\PY{p}{[}\PY{l+m+mi}{1}\PY{p}{]}\PY{p}{)}
\end{Verbatim}


    \hypertarget{optimization-functions}{%
\subsubsection{Optimization Functions}\label{optimization-functions}}

    For this implementation of BSS we are assuming no noise, and therefore
we can represent the unmixing process by the formula\\
\(x = As\)\\
where x is the mixed soundtracks, A is the mixing matrix and s is the
original sources/signals. Thus unmixing the signals/sources is as simple
as matrix multiplication with the inverse of the mixing matrix:\\
\(s = A^{-1}x\) The algorithm functions by estimating A and updating it
in the direction of our estimate's gradient. The gradient is calculated
according to the equation\\
\(gradA = -A(zs^{T} + I)\)\\
where \(s^{T}\) is the transpose of our signal matrix and z is the
matrix composed of \(z_{i}=\frac{d}{ds}log((P(s_{i}))\).\\
Since this is equivalent to
\(\frac{d}{ds}log(e^{|\frac{-s_{i}}{\lambda}|^{q}}) = -|\frac{s_{i}}{\lambda}|^{q-1}\),\\
and q=1 since we are assuming a Laplacian distribution so
\(z = -sign(s)\)

    \begin{Verbatim}[commandchars=\\\{\}]
{\color{incolor}In [{\color{incolor}7}]:} \PY{k+kn}{from} \PY{n+nn}{bsslib} \PY{k}{import} \PY{n}{bss}
\end{Verbatim}


    \hypertarget{verifying-on-synthetic-data}{%
\subsubsection{Verifying on Synthetic
Data}\label{verifying-on-synthetic-data}}

    First, we verify that the algorithm works by applying it to random data.
Note that as the algorithm assumes a Laplacian distribution, the test
data must also follow this distribution and this is taken into account
when generating the matrix. Also, every few iterations of the gradient
descent the norm and mixing matrix is printed to show its convergence.
Since the gradient of the mixing matrix is indicative of the slope of
the problem space, the norm of the gradient is indicative of the error
of our mixing matrix estimation. This is useful both for a stopping
criterion, as the algorithm halts once the gradient's norm falls below a
certain threshold, and for indicating the progress of convergence to the
ground truth mixing matrix. Thus, in addition to being printed every few
iterations, I chose to plot its decrease over time to visually indicate
convergence.

    \begin{Verbatim}[commandchars=\\\{\}]
{\color{incolor}In [{\color{incolor}8}]:} \PY{k+kn}{from} \PY{n+nn}{bsslib} \PY{k}{import} \PY{n}{syntheticDataGenerate}
        
        \PY{c+c1}{\PYZsh{} quantity of data points}
        \PY{n}{nsamples} \PY{o}{=} \PY{l+m+mi}{10000}
        \PY{c+c1}{\PYZsh{} specific mixing matrix (for illustration purpose)}
        \PY{n}{verifyA} \PY{o}{=} \PY{n}{np}\PY{o}{.}\PY{n}{asarray}\PY{p}{(}\PY{p}{[}\PY{p}{[}\PY{o}{\PYZhy{}}\PY{l+m+mi}{1}\PY{p}{,} \PY{l+m+mi}{1}\PY{p}{]}\PY{p}{,}\PY{p}{[}\PY{l+m+mi}{2}\PY{p}{,} \PY{l+m+mi}{2}\PY{p}{]}\PY{p}{]}\PY{p}{)}
        \PY{c+c1}{\PYZsh{} generate synthetic data}
        \PY{n}{synthData} \PY{o}{=} \PY{n}{syntheticDataGenerate}\PY{p}{(}\PY{n}{verifyA}\PY{p}{,} \PY{n}{nsamples}\PY{p}{)}
        \PY{c+c1}{\PYZsh{} do optimization with bss function}
        \PY{n}{estimateA}\PY{p}{,} \PY{n}{recoverData} \PY{o}{=} \PY{n}{bss}\PY{p}{(}\PY{n}{synthData}\PY{p}{)}
\end{Verbatim}


    \begin{Verbatim}[commandchars=\\\{\}]
gradient norm:  33485.4760590088
mixing matrix:  [[1494.72503309   -7.05158633]
 [ -19.28581994 2997.85417176]]
gradient norm:  4076.6946247128776
mixing matrix:  [[13299.74474315   280.15564425]
 [ -732.72397966 26676.60895409]]
gradient norm:  739.9705340028148
mixing matrix:  [[14719.70312361   692.33460377]
 [-1532.20678777 29519.94848529]]
gradient norm:  1023.5348743632755
mixing matrix:  [[14832.2313363   1469.36265368]
 [-2917.16593425 29734.57686374]]
gradient norm:  1871.7459719390608
mixing matrix:  [[14644.18382335  2716.67248426]
 [-5368.0318022  29362.97279961]]
gradient norm:  3184.1789532124967
mixing matrix:  [[13971.16471369  4819.11966865]
 [-9691.07780525 28029.73790351]]
gradient norm:  3266.7612703687564
mixing matrix:  [[ 12436.88275045   7473.52272688]
 [-15093.40599613  24940.03439232]]
gradient norm:  1806.36229861336
mixing matrix:  [[ 10722.76186873   9307.41371007]
 [-18870.79274569  21566.36473934]]
gradient norm:  465.1751406177683
mixing matrix:  [[ 10066.64154538   9885.13954801]
 [-19974.00983932  20173.24801534]]

    \end{Verbatim}

    \begin{center}
    \adjustimage{max size={0.9\linewidth}{0.9\paperheight}}{output_20_1.png}
    \end{center}
    { \hspace*{\fill} \\}
    
    And note how well our estimated mixing matrix converges to the ground
truth value as compared below:

    \begin{Verbatim}[commandchars=\\\{\}]
{\color{incolor}In [{\color{incolor}9}]:} \PY{k}{def} \PY{n+nf}{compareMixingMatrix}\PY{p}{(}\PY{n}{data}\PY{p}{,} \PY{n}{matA}\PY{p}{,} \PY{n}{matB}\PY{p}{)}\PY{p}{:}
            \PY{n}{plt}\PY{o}{.}\PY{n}{figure}\PY{p}{(}\PY{n}{figsize}\PY{o}{=}\PY{p}{(}\PY{l+m+mi}{16}\PY{p}{,} \PY{l+m+mi}{8}\PY{p}{)}\PY{p}{)}
            \PY{c+c1}{\PYZsh{} plot first mixing matrix}
            \PY{n}{plt}\PY{o}{.}\PY{n}{subplot}\PY{p}{(}\PY{l+m+mi}{1}\PY{p}{,}\PY{l+m+mi}{2}\PY{p}{,}\PY{l+m+mi}{1}\PY{p}{)}
            \PY{n}{drawDataWithMixingMatrix}\PY{p}{(}\PY{n}{data}\PY{p}{,} \PY{n}{matA}\PY{p}{)}
            \PY{c+c1}{\PYZsh{} plot first mixing matrix}
            \PY{n}{plt}\PY{o}{.}\PY{n}{subplot}\PY{p}{(}\PY{l+m+mi}{1}\PY{p}{,}\PY{l+m+mi}{2}\PY{p}{,}\PY{l+m+mi}{2}\PY{p}{)}
            \PY{n}{drawDataWithMixingMatrix}\PY{p}{(}\PY{n}{data}\PY{p}{,} \PY{n}{matB}\PY{p}{)}
            
        \PY{n}{compareMixingMatrix}\PY{p}{(}\PY{n}{synthData}\PY{p}{,} \PY{n}{verifyA}\PY{p}{,} \PY{n}{estimateA}\PY{p}{)}
\end{Verbatim}


    \begin{center}
    \adjustimage{max size={0.9\linewidth}{0.9\paperheight}}{output_22_0.png}
    \end{center}
    { \hspace*{\fill} \\}
    
    \hypertarget{applying-to-mixed-sound-tracks}{%
\subsubsection{Applying to Mixed Sound
Tracks}\label{applying-to-mixed-sound-tracks}}

    Now that we know the algorithm is accurate, we can apply it to the mixed
sound tracks we generated earlier and write the separated tracks to
their respective audio files:

    \begin{Verbatim}[commandchars=\\\{\}]
{\color{incolor}In [{\color{incolor}10}]:} \PY{n}{A}\PY{p}{,} \PY{n}{S} \PY{o}{=} \PY{n}{bss}\PY{p}{(}\PY{n}{X}\PY{p}{)}
\end{Verbatim}


    \begin{Verbatim}[commandchars=\\\{\}]
gradient norm:  7781.947481233633
mixing matrix:  [[358.1548549  402.05667333]
 [284.74812641 486.07361459]]
gradient norm:  731.8424490997927
mixing matrix:  [[2005.3891946  2288.35546743]
 [ 312.11339454 4295.56693667]]
gradient norm:  92.31761400946566
mixing matrix:  [[2275.20097733 2489.17132452]
 [ 437.83848409 4753.28069187]]

    \end{Verbatim}

    \begin{center}
    \adjustimage{max size={0.9\linewidth}{0.9\paperheight}}{output_25_1.png}
    \end{center}
    { \hspace*{\fill} \\}
    
    \begin{Verbatim}[commandchars=\\\{\}]
{\color{incolor}In [{\color{incolor}11}]:} \PY{c+c1}{\PYZsh{} normalized sound tracks}
         \PY{n}{S} \PY{o}{=} \PY{n}{audionorm}\PY{p}{(}\PY{n}{S}\PY{p}{)}
         \PY{c+c1}{\PYZsh{} write recovered sound track into WAV files}
         \PY{n}{wavfile}\PY{o}{.}\PY{n}{write}\PY{p}{(}\PY{l+s+s1}{\PYZsq{}}\PY{l+s+s1}{data/separatedTrackA.wav}\PY{l+s+s1}{\PYZsq{}}\PY{p}{,} \PY{l+m+mi}{22050}\PY{p}{,} \PY{n}{S}\PY{p}{[}\PY{l+m+mi}{0}\PY{p}{]}\PY{p}{)}
         \PY{n}{wavfile}\PY{o}{.}\PY{n}{write}\PY{p}{(}\PY{l+s+s1}{\PYZsq{}}\PY{l+s+s1}{data/separatedTrackB.wav}\PY{l+s+s1}{\PYZsq{}}\PY{p}{,} \PY{l+m+mi}{22050}\PY{p}{,} \PY{n}{S}\PY{p}{[}\PY{l+m+mi}{1}\PY{p}{]}\PY{p}{)}
\end{Verbatim}


    \hypertarget{evaluating-the-results}{%
\subsubsection{Evaluating the Results}\label{evaluating-the-results}}

    We can then compare our estimated mixing matrix to the ground truth:

    \begin{Verbatim}[commandchars=\\\{\}]
{\color{incolor}In [{\color{incolor}12}]:} \PY{n}{compareMixingMatrix}\PY{p}{(}\PY{n}{X}\PY{p}{,} \PY{n}{gtruthA}\PY{p}{,} \PY{n}{A}\PY{p}{)}
\end{Verbatim}


    \begin{center}
    \adjustimage{max size={0.9\linewidth}{0.9\paperheight}}{output_29_0.png}
    \end{center}
    { \hspace*{\fill} \\}
    
    \begin{Verbatim}[commandchars=\\\{\}]
{\color{incolor}In [{\color{incolor}13}]:} \PY{n+nb}{print}\PY{p}{(}\PY{l+s+s1}{\PYZsq{}}\PY{l+s+se}{\PYZbs{}n}\PY{l+s+s1}{Mixing Matrix (Our Estimation)}\PY{l+s+se}{\PYZbs{}n}\PY{l+s+se}{\PYZbs{}n}\PY{l+s+s1}{\PYZsq{}}\PY{p}{,} \PY{n}{A}\PY{p}{)}
         \PY{n+nb}{print}\PY{p}{(}\PY{l+s+s1}{\PYZsq{}}\PY{l+s+se}{\PYZbs{}n}\PY{l+s+s1}{Mixing Matrix (Groud Truth)}\PY{l+s+se}{\PYZbs{}n}\PY{l+s+se}{\PYZbs{}n}\PY{l+s+s1}{\PYZsq{}}\PY{p}{,} \PY{n}{gtruthA}\PY{p}{)}
\end{Verbatim}


    \begin{Verbatim}[commandchars=\\\{\}]

Mixing Matrix (Our Estimation)

 [[2275.20097733 2489.17132452]
 [ 437.83848409 4753.28069187]]

Mixing Matrix (Groud Truth)

 [[0.21815098 0.37051441]
 [0.42621966 0.06762279]]

    \end{Verbatim}

    Notice how our mixing matrix estimation is drastically different from
the ground truth. Notice, though, that the basis vectors are simply
flipped, meaning that although they numerically appear very different,
the coordinate axes representing our data transformation are still
equivalent and so the mixing matrix will yield the same result as with
the ground truth.

    \hypertarget{further-exploration-limited-sampling}{%
\subsubsection{Further Exploration: Limited
Sampling}\label{further-exploration-limited-sampling}}

    Now let us suppose that iteratively calculating the gradient for the
mixing matrix is computationally intractable. We can try to remedy this
by taking limited samples of our mixed soundtrack data and optimizing
the mixing matrix based on those. Since we are not making any
assumptions about our signals, the only way we can avoid taking unbiased
samples is to perform random sampling:

    \begin{Verbatim}[commandchars=\\\{\}]
{\color{incolor}In [{\color{incolor}14}]:} \PY{k+kn}{import} \PY{n+nn}{random}
         \PY{n}{inversesamplingrate} \PY{o}{=} \PY{l+m+mi}{50} \PY{c+c1}{\PYZsh{} randomly take 1 out of every 50 samples}
         \PY{n}{sampledXT} \PY{o}{=} \PY{p}{[}\PY{p}{]}
         \PY{k}{for} \PY{n}{col} \PY{o+ow}{in} \PY{n}{np}\PY{o}{.}\PY{n}{transpose}\PY{p}{(}\PY{n}{X}\PY{p}{)}\PY{p}{:}
             \PY{k}{if} \PY{n}{random}\PY{o}{.}\PY{n}{randint}\PY{p}{(}\PY{l+m+mi}{1}\PY{p}{,}\PY{n}{inversesamplingrate}\PY{p}{)} \PY{o}{==} \PY{l+m+mi}{1}\PY{p}{:}
                 \PY{n}{sampledXT}\PY{o}{.}\PY{n}{append}\PY{p}{(}\PY{n}{col}\PY{p}{)}
         \PY{n}{sampledX} \PY{o}{=} \PY{n}{np}\PY{o}{.}\PY{n}{transpose}\PY{p}{(}\PY{n}{sampledXT}\PY{p}{)}
         \PY{n}{sampledA}\PY{p}{,} \PY{n}{sampledS} \PY{o}{=} \PY{n}{bss}\PY{p}{(}\PY{n}{sampledX}\PY{p}{)}
         \PY{n}{compareMixingMatrix}\PY{p}{(}\PY{n}{sampledX}\PY{p}{,} \PY{n}{gtruthA}\PY{p}{,} \PY{n}{sampledA}\PY{p}{)}
         \PY{c+c1}{\PYZsh{} normalized sound tracks}
         \PY{n}{sampledS} \PY{o}{=} \PY{n}{audionorm}\PY{p}{(}\PY{n}{sampledS}\PY{p}{)}
         \PY{c+c1}{\PYZsh{} write recovered sound track into WAV files}
         \PY{n}{wavfile}\PY{o}{.}\PY{n}{write}\PY{p}{(}\PY{l+s+s1}{\PYZsq{}}\PY{l+s+s1}{data/separatedsampledTrackA.wav}\PY{l+s+s1}{\PYZsq{}}\PY{p}{,} \PY{l+m+mi}{22050}\PY{p}{,} \PY{n}{S}\PY{p}{[}\PY{l+m+mi}{0}\PY{p}{]}\PY{p}{)}
         \PY{n}{wavfile}\PY{o}{.}\PY{n}{write}\PY{p}{(}\PY{l+s+s1}{\PYZsq{}}\PY{l+s+s1}{data/separatedsampledTrackB.wav}\PY{l+s+s1}{\PYZsq{}}\PY{p}{,} \PY{l+m+mi}{22050}\PY{p}{,} \PY{n}{S}\PY{p}{[}\PY{l+m+mi}{1}\PY{p}{]}\PY{p}{)}
\end{Verbatim}


    \begin{Verbatim}[commandchars=\\\{\}]
gradient norm:  160.10278206544254
mixing matrix:  [[ 8.38017632  8.37091382]
 [ 5.93982347 10.82386404]]

    \end{Verbatim}

    \begin{center}
    \adjustimage{max size={0.9\linewidth}{0.9\paperheight}}{output_34_1.png}
    \end{center}
    { \hspace*{\fill} \\}
    
    \begin{center}
    \adjustimage{max size={0.9\linewidth}{0.9\paperheight}}{output_34_2.png}
    \end{center}
    { \hspace*{\fill} \\}
    
    Using the sampled data is much more efficient and noticably reduces
compute time, but since we run until a convergence threshold is
achieved, the results still sound the same and are on the same order of
accuracy relative to our estimated matrix, and the quality of the sound
files indicates minimal distance from the ground truth. Even with only
1/50 of the original datapoints, we can still accurately estimate the
mixing matrix in a small fraction of the original time.


    % Add a bibliography block to the postdoc
    
    
    
    \end{document}
