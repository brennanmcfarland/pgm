
% Default to the notebook output style

    


% Inherit from the specified cell style.




    
\documentclass[11pt]{article}

    
    
    \usepackage[T1]{fontenc}
    % Nicer default font (+ math font) than Computer Modern for most use cases
    \usepackage{mathpazo}

    % Basic figure setup, for now with no caption control since it's done
    % automatically by Pandoc (which extracts ![](path) syntax from Markdown).
    \usepackage{graphicx}
    % We will generate all images so they have a width \maxwidth. This means
    % that they will get their normal width if they fit onto the page, but
    % are scaled down if they would overflow the margins.
    \makeatletter
    \def\maxwidth{\ifdim\Gin@nat@width>\linewidth\linewidth
    \else\Gin@nat@width\fi}
    \makeatother
    \let\Oldincludegraphics\includegraphics
    % Set max figure width to be 80% of text width, for now hardcoded.
    \renewcommand{\includegraphics}[1]{\Oldincludegraphics[width=.8\maxwidth]{#1}}
    % Ensure that by default, figures have no caption (until we provide a
    % proper Figure object with a Caption API and a way to capture that
    % in the conversion process - todo).
    \usepackage{caption}
    \DeclareCaptionLabelFormat{nolabel}{}
    \captionsetup{labelformat=nolabel}

    \usepackage{adjustbox} % Used to constrain images to a maximum size 
    \usepackage{xcolor} % Allow colors to be defined
    \usepackage{enumerate} % Needed for markdown enumerations to work
    \usepackage{geometry} % Used to adjust the document margins
    \usepackage{amsmath} % Equations
    \usepackage{amssymb} % Equations
    \usepackage{textcomp} % defines textquotesingle
    % Hack from http://tex.stackexchange.com/a/47451/13684:
    \AtBeginDocument{%
        \def\PYZsq{\textquotesingle}% Upright quotes in Pygmentized code
    }
    \usepackage{upquote} % Upright quotes for verbatim code
    \usepackage{eurosym} % defines \euro
    \usepackage[mathletters]{ucs} % Extended unicode (utf-8) support
    \usepackage[utf8x]{inputenc} % Allow utf-8 characters in the tex document
    \usepackage{fancyvrb} % verbatim replacement that allows latex
    \usepackage{grffile} % extends the file name processing of package graphics 
                         % to support a larger range 
    % The hyperref package gives us a pdf with properly built
    % internal navigation ('pdf bookmarks' for the table of contents,
    % internal cross-reference links, web links for URLs, etc.)
    \usepackage{hyperref}
    \usepackage{longtable} % longtable support required by pandoc >1.10
    \usepackage{booktabs}  % table support for pandoc > 1.12.2
    \usepackage[inline]{enumitem} % IRkernel/repr support (it uses the enumerate* environment)
    \usepackage[normalem]{ulem} % ulem is needed to support strikethroughs (\sout)
                                % normalem makes italics be italics, not underlines
    

    
    
    % Colors for the hyperref package
    \definecolor{urlcolor}{rgb}{0,.145,.698}
    \definecolor{linkcolor}{rgb}{.71,0.21,0.01}
    \definecolor{citecolor}{rgb}{.12,.54,.11}

    % ANSI colors
    \definecolor{ansi-black}{HTML}{3E424D}
    \definecolor{ansi-black-intense}{HTML}{282C36}
    \definecolor{ansi-red}{HTML}{E75C58}
    \definecolor{ansi-red-intense}{HTML}{B22B31}
    \definecolor{ansi-green}{HTML}{00A250}
    \definecolor{ansi-green-intense}{HTML}{007427}
    \definecolor{ansi-yellow}{HTML}{DDB62B}
    \definecolor{ansi-yellow-intense}{HTML}{B27D12}
    \definecolor{ansi-blue}{HTML}{208FFB}
    \definecolor{ansi-blue-intense}{HTML}{0065CA}
    \definecolor{ansi-magenta}{HTML}{D160C4}
    \definecolor{ansi-magenta-intense}{HTML}{A03196}
    \definecolor{ansi-cyan}{HTML}{60C6C8}
    \definecolor{ansi-cyan-intense}{HTML}{258F8F}
    \definecolor{ansi-white}{HTML}{C5C1B4}
    \definecolor{ansi-white-intense}{HTML}{A1A6B2}

    % commands and environments needed by pandoc snippets
    % extracted from the output of `pandoc -s`
    \providecommand{\tightlist}{%
      \setlength{\itemsep}{0pt}\setlength{\parskip}{0pt}}
    \DefineVerbatimEnvironment{Highlighting}{Verbatim}{commandchars=\\\{\}}
    % Add ',fontsize=\small' for more characters per line
    \newenvironment{Shaded}{}{}
    \newcommand{\KeywordTok}[1]{\textcolor[rgb]{0.00,0.44,0.13}{\textbf{{#1}}}}
    \newcommand{\DataTypeTok}[1]{\textcolor[rgb]{0.56,0.13,0.00}{{#1}}}
    \newcommand{\DecValTok}[1]{\textcolor[rgb]{0.25,0.63,0.44}{{#1}}}
    \newcommand{\BaseNTok}[1]{\textcolor[rgb]{0.25,0.63,0.44}{{#1}}}
    \newcommand{\FloatTok}[1]{\textcolor[rgb]{0.25,0.63,0.44}{{#1}}}
    \newcommand{\CharTok}[1]{\textcolor[rgb]{0.25,0.44,0.63}{{#1}}}
    \newcommand{\StringTok}[1]{\textcolor[rgb]{0.25,0.44,0.63}{{#1}}}
    \newcommand{\CommentTok}[1]{\textcolor[rgb]{0.38,0.63,0.69}{\textit{{#1}}}}
    \newcommand{\OtherTok}[1]{\textcolor[rgb]{0.00,0.44,0.13}{{#1}}}
    \newcommand{\AlertTok}[1]{\textcolor[rgb]{1.00,0.00,0.00}{\textbf{{#1}}}}
    \newcommand{\FunctionTok}[1]{\textcolor[rgb]{0.02,0.16,0.49}{{#1}}}
    \newcommand{\RegionMarkerTok}[1]{{#1}}
    \newcommand{\ErrorTok}[1]{\textcolor[rgb]{1.00,0.00,0.00}{\textbf{{#1}}}}
    \newcommand{\NormalTok}[1]{{#1}}
    
    % Additional commands for more recent versions of Pandoc
    \newcommand{\ConstantTok}[1]{\textcolor[rgb]{0.53,0.00,0.00}{{#1}}}
    \newcommand{\SpecialCharTok}[1]{\textcolor[rgb]{0.25,0.44,0.63}{{#1}}}
    \newcommand{\VerbatimStringTok}[1]{\textcolor[rgb]{0.25,0.44,0.63}{{#1}}}
    \newcommand{\SpecialStringTok}[1]{\textcolor[rgb]{0.73,0.40,0.53}{{#1}}}
    \newcommand{\ImportTok}[1]{{#1}}
    \newcommand{\DocumentationTok}[1]{\textcolor[rgb]{0.73,0.13,0.13}{\textit{{#1}}}}
    \newcommand{\AnnotationTok}[1]{\textcolor[rgb]{0.38,0.63,0.69}{\textbf{\textit{{#1}}}}}
    \newcommand{\CommentVarTok}[1]{\textcolor[rgb]{0.38,0.63,0.69}{\textbf{\textit{{#1}}}}}
    \newcommand{\VariableTok}[1]{\textcolor[rgb]{0.10,0.09,0.49}{{#1}}}
    \newcommand{\ControlFlowTok}[1]{\textcolor[rgb]{0.00,0.44,0.13}{\textbf{{#1}}}}
    \newcommand{\OperatorTok}[1]{\textcolor[rgb]{0.40,0.40,0.40}{{#1}}}
    \newcommand{\BuiltInTok}[1]{{#1}}
    \newcommand{\ExtensionTok}[1]{{#1}}
    \newcommand{\PreprocessorTok}[1]{\textcolor[rgb]{0.74,0.48,0.00}{{#1}}}
    \newcommand{\AttributeTok}[1]{\textcolor[rgb]{0.49,0.56,0.16}{{#1}}}
    \newcommand{\InformationTok}[1]{\textcolor[rgb]{0.38,0.63,0.69}{\textbf{\textit{{#1}}}}}
    \newcommand{\WarningTok}[1]{\textcolor[rgb]{0.38,0.63,0.69}{\textbf{\textit{{#1}}}}}
    
    
    % Define a nice break command that doesn't care if a line doesn't already
    % exist.
    \def\br{\hspace*{\fill} \\* }
    % Math Jax compatability definitions
    \def\gt{>}
    \def\lt{<}
    % Document parameters
    \title{Homework 3}
    
    
    

    % Pygments definitions
    
\makeatletter
\def\PY@reset{\let\PY@it=\relax \let\PY@bf=\relax%
    \let\PY@ul=\relax \let\PY@tc=\relax%
    \let\PY@bc=\relax \let\PY@ff=\relax}
\def\PY@tok#1{\csname PY@tok@#1\endcsname}
\def\PY@toks#1+{\ifx\relax#1\empty\else%
    \PY@tok{#1}\expandafter\PY@toks\fi}
\def\PY@do#1{\PY@bc{\PY@tc{\PY@ul{%
    \PY@it{\PY@bf{\PY@ff{#1}}}}}}}
\def\PY#1#2{\PY@reset\PY@toks#1+\relax+\PY@do{#2}}

\expandafter\def\csname PY@tok@w\endcsname{\def\PY@tc##1{\textcolor[rgb]{0.73,0.73,0.73}{##1}}}
\expandafter\def\csname PY@tok@c\endcsname{\let\PY@it=\textit\def\PY@tc##1{\textcolor[rgb]{0.25,0.50,0.50}{##1}}}
\expandafter\def\csname PY@tok@cp\endcsname{\def\PY@tc##1{\textcolor[rgb]{0.74,0.48,0.00}{##1}}}
\expandafter\def\csname PY@tok@k\endcsname{\let\PY@bf=\textbf\def\PY@tc##1{\textcolor[rgb]{0.00,0.50,0.00}{##1}}}
\expandafter\def\csname PY@tok@kp\endcsname{\def\PY@tc##1{\textcolor[rgb]{0.00,0.50,0.00}{##1}}}
\expandafter\def\csname PY@tok@kt\endcsname{\def\PY@tc##1{\textcolor[rgb]{0.69,0.00,0.25}{##1}}}
\expandafter\def\csname PY@tok@o\endcsname{\def\PY@tc##1{\textcolor[rgb]{0.40,0.40,0.40}{##1}}}
\expandafter\def\csname PY@tok@ow\endcsname{\let\PY@bf=\textbf\def\PY@tc##1{\textcolor[rgb]{0.67,0.13,1.00}{##1}}}
\expandafter\def\csname PY@tok@nb\endcsname{\def\PY@tc##1{\textcolor[rgb]{0.00,0.50,0.00}{##1}}}
\expandafter\def\csname PY@tok@nf\endcsname{\def\PY@tc##1{\textcolor[rgb]{0.00,0.00,1.00}{##1}}}
\expandafter\def\csname PY@tok@nc\endcsname{\let\PY@bf=\textbf\def\PY@tc##1{\textcolor[rgb]{0.00,0.00,1.00}{##1}}}
\expandafter\def\csname PY@tok@nn\endcsname{\let\PY@bf=\textbf\def\PY@tc##1{\textcolor[rgb]{0.00,0.00,1.00}{##1}}}
\expandafter\def\csname PY@tok@ne\endcsname{\let\PY@bf=\textbf\def\PY@tc##1{\textcolor[rgb]{0.82,0.25,0.23}{##1}}}
\expandafter\def\csname PY@tok@nv\endcsname{\def\PY@tc##1{\textcolor[rgb]{0.10,0.09,0.49}{##1}}}
\expandafter\def\csname PY@tok@no\endcsname{\def\PY@tc##1{\textcolor[rgb]{0.53,0.00,0.00}{##1}}}
\expandafter\def\csname PY@tok@nl\endcsname{\def\PY@tc##1{\textcolor[rgb]{0.63,0.63,0.00}{##1}}}
\expandafter\def\csname PY@tok@ni\endcsname{\let\PY@bf=\textbf\def\PY@tc##1{\textcolor[rgb]{0.60,0.60,0.60}{##1}}}
\expandafter\def\csname PY@tok@na\endcsname{\def\PY@tc##1{\textcolor[rgb]{0.49,0.56,0.16}{##1}}}
\expandafter\def\csname PY@tok@nt\endcsname{\let\PY@bf=\textbf\def\PY@tc##1{\textcolor[rgb]{0.00,0.50,0.00}{##1}}}
\expandafter\def\csname PY@tok@nd\endcsname{\def\PY@tc##1{\textcolor[rgb]{0.67,0.13,1.00}{##1}}}
\expandafter\def\csname PY@tok@s\endcsname{\def\PY@tc##1{\textcolor[rgb]{0.73,0.13,0.13}{##1}}}
\expandafter\def\csname PY@tok@sd\endcsname{\let\PY@it=\textit\def\PY@tc##1{\textcolor[rgb]{0.73,0.13,0.13}{##1}}}
\expandafter\def\csname PY@tok@si\endcsname{\let\PY@bf=\textbf\def\PY@tc##1{\textcolor[rgb]{0.73,0.40,0.53}{##1}}}
\expandafter\def\csname PY@tok@se\endcsname{\let\PY@bf=\textbf\def\PY@tc##1{\textcolor[rgb]{0.73,0.40,0.13}{##1}}}
\expandafter\def\csname PY@tok@sr\endcsname{\def\PY@tc##1{\textcolor[rgb]{0.73,0.40,0.53}{##1}}}
\expandafter\def\csname PY@tok@ss\endcsname{\def\PY@tc##1{\textcolor[rgb]{0.10,0.09,0.49}{##1}}}
\expandafter\def\csname PY@tok@sx\endcsname{\def\PY@tc##1{\textcolor[rgb]{0.00,0.50,0.00}{##1}}}
\expandafter\def\csname PY@tok@m\endcsname{\def\PY@tc##1{\textcolor[rgb]{0.40,0.40,0.40}{##1}}}
\expandafter\def\csname PY@tok@gh\endcsname{\let\PY@bf=\textbf\def\PY@tc##1{\textcolor[rgb]{0.00,0.00,0.50}{##1}}}
\expandafter\def\csname PY@tok@gu\endcsname{\let\PY@bf=\textbf\def\PY@tc##1{\textcolor[rgb]{0.50,0.00,0.50}{##1}}}
\expandafter\def\csname PY@tok@gd\endcsname{\def\PY@tc##1{\textcolor[rgb]{0.63,0.00,0.00}{##1}}}
\expandafter\def\csname PY@tok@gi\endcsname{\def\PY@tc##1{\textcolor[rgb]{0.00,0.63,0.00}{##1}}}
\expandafter\def\csname PY@tok@gr\endcsname{\def\PY@tc##1{\textcolor[rgb]{1.00,0.00,0.00}{##1}}}
\expandafter\def\csname PY@tok@ge\endcsname{\let\PY@it=\textit}
\expandafter\def\csname PY@tok@gs\endcsname{\let\PY@bf=\textbf}
\expandafter\def\csname PY@tok@gp\endcsname{\let\PY@bf=\textbf\def\PY@tc##1{\textcolor[rgb]{0.00,0.00,0.50}{##1}}}
\expandafter\def\csname PY@tok@go\endcsname{\def\PY@tc##1{\textcolor[rgb]{0.53,0.53,0.53}{##1}}}
\expandafter\def\csname PY@tok@gt\endcsname{\def\PY@tc##1{\textcolor[rgb]{0.00,0.27,0.87}{##1}}}
\expandafter\def\csname PY@tok@err\endcsname{\def\PY@bc##1{\setlength{\fboxsep}{0pt}\fcolorbox[rgb]{1.00,0.00,0.00}{1,1,1}{\strut ##1}}}
\expandafter\def\csname PY@tok@kc\endcsname{\let\PY@bf=\textbf\def\PY@tc##1{\textcolor[rgb]{0.00,0.50,0.00}{##1}}}
\expandafter\def\csname PY@tok@kd\endcsname{\let\PY@bf=\textbf\def\PY@tc##1{\textcolor[rgb]{0.00,0.50,0.00}{##1}}}
\expandafter\def\csname PY@tok@kn\endcsname{\let\PY@bf=\textbf\def\PY@tc##1{\textcolor[rgb]{0.00,0.50,0.00}{##1}}}
\expandafter\def\csname PY@tok@kr\endcsname{\let\PY@bf=\textbf\def\PY@tc##1{\textcolor[rgb]{0.00,0.50,0.00}{##1}}}
\expandafter\def\csname PY@tok@bp\endcsname{\def\PY@tc##1{\textcolor[rgb]{0.00,0.50,0.00}{##1}}}
\expandafter\def\csname PY@tok@fm\endcsname{\def\PY@tc##1{\textcolor[rgb]{0.00,0.00,1.00}{##1}}}
\expandafter\def\csname PY@tok@vc\endcsname{\def\PY@tc##1{\textcolor[rgb]{0.10,0.09,0.49}{##1}}}
\expandafter\def\csname PY@tok@vg\endcsname{\def\PY@tc##1{\textcolor[rgb]{0.10,0.09,0.49}{##1}}}
\expandafter\def\csname PY@tok@vi\endcsname{\def\PY@tc##1{\textcolor[rgb]{0.10,0.09,0.49}{##1}}}
\expandafter\def\csname PY@tok@vm\endcsname{\def\PY@tc##1{\textcolor[rgb]{0.10,0.09,0.49}{##1}}}
\expandafter\def\csname PY@tok@sa\endcsname{\def\PY@tc##1{\textcolor[rgb]{0.73,0.13,0.13}{##1}}}
\expandafter\def\csname PY@tok@sb\endcsname{\def\PY@tc##1{\textcolor[rgb]{0.73,0.13,0.13}{##1}}}
\expandafter\def\csname PY@tok@sc\endcsname{\def\PY@tc##1{\textcolor[rgb]{0.73,0.13,0.13}{##1}}}
\expandafter\def\csname PY@tok@dl\endcsname{\def\PY@tc##1{\textcolor[rgb]{0.73,0.13,0.13}{##1}}}
\expandafter\def\csname PY@tok@s2\endcsname{\def\PY@tc##1{\textcolor[rgb]{0.73,0.13,0.13}{##1}}}
\expandafter\def\csname PY@tok@sh\endcsname{\def\PY@tc##1{\textcolor[rgb]{0.73,0.13,0.13}{##1}}}
\expandafter\def\csname PY@tok@s1\endcsname{\def\PY@tc##1{\textcolor[rgb]{0.73,0.13,0.13}{##1}}}
\expandafter\def\csname PY@tok@mb\endcsname{\def\PY@tc##1{\textcolor[rgb]{0.40,0.40,0.40}{##1}}}
\expandafter\def\csname PY@tok@mf\endcsname{\def\PY@tc##1{\textcolor[rgb]{0.40,0.40,0.40}{##1}}}
\expandafter\def\csname PY@tok@mh\endcsname{\def\PY@tc##1{\textcolor[rgb]{0.40,0.40,0.40}{##1}}}
\expandafter\def\csname PY@tok@mi\endcsname{\def\PY@tc##1{\textcolor[rgb]{0.40,0.40,0.40}{##1}}}
\expandafter\def\csname PY@tok@il\endcsname{\def\PY@tc##1{\textcolor[rgb]{0.40,0.40,0.40}{##1}}}
\expandafter\def\csname PY@tok@mo\endcsname{\def\PY@tc##1{\textcolor[rgb]{0.40,0.40,0.40}{##1}}}
\expandafter\def\csname PY@tok@ch\endcsname{\let\PY@it=\textit\def\PY@tc##1{\textcolor[rgb]{0.25,0.50,0.50}{##1}}}
\expandafter\def\csname PY@tok@cm\endcsname{\let\PY@it=\textit\def\PY@tc##1{\textcolor[rgb]{0.25,0.50,0.50}{##1}}}
\expandafter\def\csname PY@tok@cpf\endcsname{\let\PY@it=\textit\def\PY@tc##1{\textcolor[rgb]{0.25,0.50,0.50}{##1}}}
\expandafter\def\csname PY@tok@c1\endcsname{\let\PY@it=\textit\def\PY@tc##1{\textcolor[rgb]{0.25,0.50,0.50}{##1}}}
\expandafter\def\csname PY@tok@cs\endcsname{\let\PY@it=\textit\def\PY@tc##1{\textcolor[rgb]{0.25,0.50,0.50}{##1}}}

\def\PYZbs{\char`\\}
\def\PYZus{\char`\_}
\def\PYZob{\char`\{}
\def\PYZcb{\char`\}}
\def\PYZca{\char`\^}
\def\PYZam{\char`\&}
\def\PYZlt{\char`\<}
\def\PYZgt{\char`\>}
\def\PYZsh{\char`\#}
\def\PYZpc{\char`\%}
\def\PYZdl{\char`\$}
\def\PYZhy{\char`\-}
\def\PYZsq{\char`\'}
\def\PYZdq{\char`\"}
\def\PYZti{\char`\~}
% for compatibility with earlier versions
\def\PYZat{@}
\def\PYZlb{[}
\def\PYZrb{]}
\makeatother


    % Exact colors from NB
    \definecolor{incolor}{rgb}{0.0, 0.0, 0.5}
    \definecolor{outcolor}{rgb}{0.545, 0.0, 0.0}



    
    % Prevent overflowing lines due to hard-to-break entities
    \sloppy 
    % Setup hyperref package
    \hypersetup{
      breaklinks=true,  % so long urls are correctly broken across lines
      colorlinks=true,
      urlcolor=urlcolor,
      linkcolor=linkcolor,
      citecolor=citecolor,
      }
    % Slightly bigger margins than the latex defaults
    
    \geometry{verbose,tmargin=1in,bmargin=1in,lmargin=1in,rmargin=1in}
    
    

    \begin{document}
    
    
    \maketitle
    
    

    
    \hypertarget{homework-3---larger-graphical-model}{%
\section{Homework 3 - Larger Graphical
Model}\label{homework-3---larger-graphical-model}}

    Brennan McFarland\\
bfm21

    \hypertarget{problem-description}{%
\subsection{Problem Description}\label{problem-description}}

    Suppose we want to monitor the success of a network packet transmission
relative to various factors affecting the reliability of the network.
This information could be used to infer the condition of the network or
troubleshoot problems based on such limited information as whether the
packet was successfully transmitted and nothing else.

    \hypertarget{background}{%
\subsection{Background}\label{background}}

    Our transmission utilizes an acknowledgement-based protocol such as TCP
where the receiver can acknowledge back to the host which packets it has
received. This is a widespread technique to increase protocol
reliability, as the sender can send a duplicate of the original packet
if it has not received the corresponding acknowledgement within a given
period of time. Since our network will be affected by the error rate in
a given transmission, we will also assume that this protocol is either
not error-correcting or that repeated error-correction is liable to
decrease the chances of successful packet transmission, for example by
requiring more data to be sent that could potentially be lost.

    \hypertarget{probability-model}{%
\subsection{Probability Model}\label{probability-model}}

    For our probability model we will take into account the following
variables:

\begin{itemize}
\tightlist
\item
  \(V\) : successful packet transmission - binary (P is for probability
  and we use S and T elsewhere)
\item
  \(D\) : number of duplicate packets sent thus far - discrete
\item
  \(A\) : time since last packet acknowledgement - discrete (measured in
  time steps)
\item
  \(E\) : packet error rate - discrete (measured in \# errors)
\item
  \(L\) : network load - discrete (measured in multiples of packets)
\item
  \(U\) : network is up - binary
\item
  \(S\) : sender is up - binary
\item
  \(R\) : receiver is up - binary
\item
  \(C\) : connection between sender and receiver is up - binary
\end{itemize}

    \begin{Verbatim}[commandchars=\\\{\}]
{\color{incolor}In [{\color{incolor}25}]:} \PY{k+kn}{from} \PY{n+nn}{graphviz} \PY{k}{import} \PY{n}{Digraph}
         \PY{k+kn}{import} \PY{n+nn}{matplotlib}\PY{n+nn}{.}\PY{n+nn}{pyplot} \PY{k}{as} \PY{n+nn}{plt}
         \PY{k+kn}{import} \PY{n+nn}{matplotlib}\PY{n+nn}{.}\PY{n+nn}{image} \PY{k}{as} \PY{n+nn}{imtool}
         
         \PY{n}{image\PYZus{}format} \PY{o}{=} \PY{l+s+s1}{\PYZsq{}}\PY{l+s+s1}{png}\PY{l+s+s1}{\PYZsq{}}
         \PY{n}{dot} \PY{o}{=} \PY{n}{Digraph}\PY{p}{(}\PY{n+nb}{format}\PY{o}{=}\PY{n}{image\PYZus{}format}\PY{p}{)}
         \PY{n}{dot}\PY{o}{.}\PY{n}{node}\PY{p}{(}\PY{l+s+s1}{\PYZsq{}}\PY{l+s+s1}{V}\PY{l+s+s1}{\PYZsq{}}\PY{p}{,} \PY{l+s+s1}{\PYZsq{}}\PY{l+s+s1}{V}\PY{l+s+s1}{\PYZsq{}}\PY{p}{)}
         \PY{n}{dot}\PY{o}{.}\PY{n}{node}\PY{p}{(}\PY{l+s+s1}{\PYZsq{}}\PY{l+s+s1}{D}\PY{l+s+s1}{\PYZsq{}}\PY{p}{,} \PY{l+s+s1}{\PYZsq{}}\PY{l+s+s1}{D}\PY{l+s+s1}{\PYZsq{}}\PY{p}{)}
         \PY{n}{dot}\PY{o}{.}\PY{n}{node}\PY{p}{(}\PY{l+s+s1}{\PYZsq{}}\PY{l+s+s1}{A}\PY{l+s+s1}{\PYZsq{}}\PY{p}{,} \PY{l+s+s1}{\PYZsq{}}\PY{l+s+s1}{A}\PY{l+s+s1}{\PYZsq{}}\PY{p}{)}
         \PY{n}{dot}\PY{o}{.}\PY{n}{node}\PY{p}{(}\PY{l+s+s1}{\PYZsq{}}\PY{l+s+s1}{E}\PY{l+s+s1}{\PYZsq{}}\PY{p}{,} \PY{l+s+s1}{\PYZsq{}}\PY{l+s+s1}{E}\PY{l+s+s1}{\PYZsq{}}\PY{p}{)}
         \PY{n}{dot}\PY{o}{.}\PY{n}{node}\PY{p}{(}\PY{l+s+s1}{\PYZsq{}}\PY{l+s+s1}{L}\PY{l+s+s1}{\PYZsq{}}\PY{p}{,} \PY{l+s+s1}{\PYZsq{}}\PY{l+s+s1}{L}\PY{l+s+s1}{\PYZsq{}}\PY{p}{)}
         \PY{n}{dot}\PY{o}{.}\PY{n}{node}\PY{p}{(}\PY{l+s+s1}{\PYZsq{}}\PY{l+s+s1}{U}\PY{l+s+s1}{\PYZsq{}}\PY{p}{,} \PY{l+s+s1}{\PYZsq{}}\PY{l+s+s1}{U}\PY{l+s+s1}{\PYZsq{}}\PY{p}{)}
         \PY{n}{dot}\PY{o}{.}\PY{n}{node}\PY{p}{(}\PY{l+s+s1}{\PYZsq{}}\PY{l+s+s1}{S}\PY{l+s+s1}{\PYZsq{}}\PY{p}{,} \PY{l+s+s1}{\PYZsq{}}\PY{l+s+s1}{S}\PY{l+s+s1}{\PYZsq{}}\PY{p}{)}
         \PY{n}{dot}\PY{o}{.}\PY{n}{node}\PY{p}{(}\PY{l+s+s1}{\PYZsq{}}\PY{l+s+s1}{R}\PY{l+s+s1}{\PYZsq{}}\PY{p}{,} \PY{l+s+s1}{\PYZsq{}}\PY{l+s+s1}{R}\PY{l+s+s1}{\PYZsq{}}\PY{p}{)}
         \PY{n}{dot}\PY{o}{.}\PY{n}{node}\PY{p}{(}\PY{l+s+s1}{\PYZsq{}}\PY{l+s+s1}{C}\PY{l+s+s1}{\PYZsq{}}\PY{p}{,} \PY{l+s+s1}{\PYZsq{}}\PY{l+s+s1}{C}\PY{l+s+s1}{\PYZsq{}}\PY{p}{)}
         \PY{n}{dot}\PY{o}{.}\PY{n}{edges}\PY{p}{(}\PY{p}{[}\PY{l+s+s1}{\PYZsq{}}\PY{l+s+s1}{DV}\PY{l+s+s1}{\PYZsq{}}\PY{p}{,} \PY{l+s+s1}{\PYZsq{}}\PY{l+s+s1}{AV}\PY{l+s+s1}{\PYZsq{}}\PY{p}{,} \PY{l+s+s1}{\PYZsq{}}\PY{l+s+s1}{EV}\PY{l+s+s1}{\PYZsq{}}\PY{p}{,} \PY{l+s+s1}{\PYZsq{}}\PY{l+s+s1}{LD}\PY{l+s+s1}{\PYZsq{}}\PY{p}{,} \PY{l+s+s1}{\PYZsq{}}\PY{l+s+s1}{LA}\PY{l+s+s1}{\PYZsq{}}\PY{p}{,} \PY{l+s+s1}{\PYZsq{}}\PY{l+s+s1}{UL}\PY{l+s+s1}{\PYZsq{}}\PY{p}{,} \PY{l+s+s1}{\PYZsq{}}\PY{l+s+s1}{UD}\PY{l+s+s1}{\PYZsq{}}\PY{p}{,} \PY{l+s+s1}{\PYZsq{}}\PY{l+s+s1}{UA}\PY{l+s+s1}{\PYZsq{}}\PY{p}{,} \PY{l+s+s1}{\PYZsq{}}\PY{l+s+s1}{CU}\PY{l+s+s1}{\PYZsq{}}\PY{p}{,} \PY{l+s+s1}{\PYZsq{}}\PY{l+s+s1}{RU}\PY{l+s+s1}{\PYZsq{}}\PY{p}{,} \PY{l+s+s1}{\PYZsq{}}\PY{l+s+s1}{SU}\PY{l+s+s1}{\PYZsq{}}\PY{p}{]}\PY{p}{)}
         \PY{n}{path} \PY{o}{=} \PY{l+s+s1}{\PYZsq{}}\PY{l+s+s1}{exercise1\PYZhy{}graph}\PY{l+s+s1}{\PYZsq{}}
         \PY{n}{dot}\PY{o}{.}\PY{n}{render}\PY{p}{(}\PY{n}{path}\PY{p}{)}
         \PY{n}{imorg} \PY{o}{=} \PY{n}{imtool}\PY{o}{.}\PY{n}{imread}\PY{p}{(}\PY{n}{path} \PY{o}{+} \PY{l+s+s1}{\PYZsq{}}\PY{l+s+s1}{.}\PY{l+s+s1}{\PYZsq{}} \PY{o}{+} \PY{n}{image\PYZus{}format}\PY{p}{)}
         \PY{n}{plt}\PY{o}{.}\PY{n}{axis}\PY{p}{(}\PY{l+s+s1}{\PYZsq{}}\PY{l+s+s1}{off}\PY{l+s+s1}{\PYZsq{}}\PY{p}{)}
         \PY{n}{plt}\PY{o}{.}\PY{n}{imshow}\PY{p}{(}\PY{n}{imorg}\PY{p}{)} \PY{c+c1}{\PYZsh{} NOTE: the first time it might not show the image, if it doesn\PYZsq{}t just execute this block again}
\end{Verbatim}


\begin{Verbatim}[commandchars=\\\{\}]
{\color{outcolor}Out[{\color{outcolor}25}]:} <matplotlib.image.AxesImage at 0x7fc26fd9c748>
\end{Verbatim}
            
    \begin{center}
    \adjustimage{max size={0.9\linewidth}{0.9\paperheight}}{output_8_1.png}
    \end{center}
    { \hspace*{\fill} \\}
    
    We define reasonable probabilities for this model when building it in
the next section.

    \hypertarget{building-the-model-in-python}{%
\subsection{Building the Model in
Python}\label{building-the-model-in-python}}

    \begin{Verbatim}[commandchars=\\\{\}]
{\color{incolor}In [{\color{incolor}2}]:} \PY{k+kn}{from} \PY{n+nn}{pgmpy}\PY{n+nn}{.}\PY{n+nn}{models} \PY{k}{import} \PY{n}{BayesianModel} \PY{k}{as} \PY{n}{bysmodel}
        \PY{k+kn}{from} \PY{n+nn}{pgmpy}\PY{n+nn}{.}\PY{n+nn}{factors}\PY{n+nn}{.}\PY{n+nn}{discrete} \PY{k}{import} \PY{n}{TabularCPD} \PY{k}{as} \PY{n}{tcpd}
        \PY{k+kn}{from} \PY{n+nn}{pgmpy}\PY{n+nn}{.}\PY{n+nn}{factors}\PY{n+nn}{.}\PY{n+nn}{continuous} \PY{k}{import} \PY{n}{ContinuousFactor}
        \PY{k+kn}{import} \PY{n+nn}{scipy}
        \PY{k+kn}{import} \PY{n+nn}{math}
        \PY{k+kn}{import} \PY{n+nn}{numpy} \PY{k}{as} \PY{n+nn}{np}
\end{Verbatim}


    Before we define our model, let us first define a few helper functions
to quickly generate the corresponding probability distributions,
beginning with the distribution functions themselves. In cases where we
are more likely to see a mean value (for example network load) with a
lesser probability of more or less we use the tried and true binomial
distribution with a mean adjusted for the range of possible values.\\
In other cases, we use a distribution whose probability is strictly
decreasing, ie, the geometric distribution:\\
\(P(k)=(1-p)^kp\)\\
Since this distribution has an infinitely large domain (positive
integers), we restrict it to the possible values of the variable. We
later normalize to make sure all probabilities sum to exactly 1.

    \begin{Verbatim}[commandchars=\\\{\}]
{\color{incolor}In [{\color{incolor}24}]:} \PY{k}{def} \PY{n+nf}{binom}\PY{p}{(}\PY{n}{w}\PY{p}{,} \PY{n}{n}\PY{p}{,} \PY{n}{p}\PY{p}{)}\PY{p}{:}
             \PY{k}{if}\PY{p}{(}\PY{n}{w} \PY{o}{\PYZlt{}} \PY{l+m+mi}{0} \PY{o+ow}{or} \PY{n}{w} \PY{o}{\PYZgt{}} \PY{n}{n}\PY{p}{)}\PY{p}{:}
                \PY{k}{return} \PY{l+m+mi}{0}
             \PY{c+c1}{\PYZsh{} it stops trying to convert to float before calculating if the numbers are too large}
             \PY{k}{try}\PY{p}{:}
                 \PY{k}{return} \PY{n+nb}{float}\PY{p}{(}\PY{n}{math}\PY{o}{.}\PY{n}{factorial}\PY{p}{(}\PY{n}{n}\PY{p}{)}\PY{p}{)}\PY{o}{/}\PY{n+nb}{float}\PY{p}{(}\PY{n}{math}\PY{o}{.}\PY{n}{factorial}\PY{p}{(}\PY{n}{n}\PY{o}{\PYZhy{}}\PY{n}{w}\PY{p}{)}\PY{o}{*}\PY{n}{math}\PY{o}{.}\PY{n}{factorial}\PY{p}{(}\PY{n}{w}\PY{p}{)}\PY{p}{)}\PY{o}{*}\PY{p}{(}\PY{n}{p}\PY{o}{*}\PY{o}{*}\PY{n}{w}\PY{p}{)}\PY{o}{*}\PY{p}{(}\PY{p}{(}\PY{l+m+mi}{1}\PY{o}{\PYZhy{}}\PY{n}{p}\PY{p}{)}\PY{o}{*}\PY{o}{*}\PY{p}{(}\PY{n}{n}\PY{o}{\PYZhy{}}\PY{n}{w}\PY{p}{)}\PY{p}{)}
             \PY{k}{except}\PY{p}{:}
                 \PY{k}{return} \PY{n}{math}\PY{o}{.}\PY{n}{factorial}\PY{p}{(}\PY{n}{n}\PY{p}{)}\PY{o}{/}\PY{p}{(}\PY{n}{math}\PY{o}{.}\PY{n}{factorial}\PY{p}{(}\PY{n}{n}\PY{o}{\PYZhy{}}\PY{n}{w}\PY{p}{)}\PY{o}{*}\PY{n}{math}\PY{o}{.}\PY{n}{factorial}\PY{p}{(}\PY{n}{w}\PY{p}{)}\PY{p}{)}\PY{o}{*}\PY{p}{(}\PY{n}{p}\PY{o}{*}\PY{o}{*}\PY{n}{w}\PY{p}{)}\PY{o}{*}\PY{p}{(}\PY{p}{(}\PY{l+m+mi}{1}\PY{o}{\PYZhy{}}\PY{n}{p}\PY{p}{)}\PY{o}{*}\PY{o}{*}\PY{p}{(}\PY{n}{n}\PY{o}{\PYZhy{}}\PY{n}{w}\PY{p}{)}\PY{p}{)}
             
         \PY{k}{def} \PY{n+nf}{geom}\PY{p}{(}\PY{n}{p}\PY{p}{,} \PY{n}{k}\PY{p}{)}\PY{p}{:}
             \PY{k}{return} \PY{p}{(}\PY{p}{(}\PY{l+m+mf}{1.0}\PY{o}{\PYZhy{}}\PY{n+nb}{float}\PY{p}{(}\PY{n}{p}\PY{p}{)}\PY{p}{)}\PY{o}{*}\PY{o}{*}\PY{n+nb}{float}\PY{p}{(}\PY{n}{k}\PY{p}{)}\PY{p}{)}\PY{o}{*}\PY{n+nb}{float}\PY{p}{(}\PY{n}{p}\PY{p}{)}
         
         \PY{k}{def} \PY{n+nf}{pdf\PYZus{}binom}\PY{p}{(}\PY{n}{n}\PY{p}{,} \PY{n}{p}\PY{p}{)}\PY{p}{:}
             \PY{k}{return} \PY{p}{[}\PY{n}{binom}\PY{p}{(}\PY{n}{i}\PY{p}{,} \PY{n}{n}\PY{p}{,} \PY{n}{p}\PY{p}{)} \PY{k}{for} \PY{n}{i} \PY{o+ow}{in} \PY{n+nb}{range}\PY{p}{(}\PY{n}{n}\PY{p}{)}\PY{p}{]}
         
         \PY{c+c1}{\PYZsh{} NOTE: since we don\PYZsq{}t want the distribution going off to infinity, we cut it off at n and normalize later for efficiency}
         \PY{k}{def} \PY{n+nf}{pdf\PYZus{}geom}\PY{p}{(}\PY{n}{p}\PY{p}{,} \PY{n}{n}\PY{p}{)}\PY{p}{:}
             \PY{k}{return} \PY{p}{[}\PY{n}{geom}\PY{p}{(}\PY{n}{p}\PY{p}{,} \PY{n}{k}\PY{p}{)} \PY{k}{for} \PY{n}{k} \PY{o+ow}{in} \PY{n+nb}{range}\PY{p}{(}\PY{n}{n}\PY{p}{)}\PY{p}{]}
         
         \PY{c+c1}{\PYZsh{} return an array for a strictly decreasing discrete distribution}
         \PY{k}{def} \PY{n+nf}{decreasing\PYZus{}distribution}\PY{p}{(}\PY{n}{numvals}\PY{p}{)}\PY{p}{:}
             \PY{n}{dist} \PY{o}{=} \PY{n}{pdf\PYZus{}geom}\PY{p}{(}\PY{l+m+mf}{1.0}\PY{o}{/}\PY{n+nb}{float}\PY{p}{(}\PY{n}{numvals}\PY{o}{*}\PY{o}{*}\PY{o}{.}\PY{l+m+mi}{5}\PY{p}{)}\PY{p}{,} \PY{n}{numvals}\PY{p}{)}
             \PY{k}{return} \PY{p}{[}\PY{n}{dist}\PY{p}{]}
         
         \PY{c+c1}{\PYZsh{} return an array for a \PYZdq{}regular\PYZdq{}, ie, increasing and then decreasing, discrete distribution}
         \PY{k}{def} \PY{n+nf}{regular\PYZus{}distribution}\PY{p}{(}\PY{n}{numvals}\PY{p}{)}\PY{p}{:}
             \PY{n}{dist} \PY{o}{=} \PY{n}{pdf\PYZus{}binom}\PY{p}{(}\PY{n}{numvals}\PY{p}{,} \PY{o}{.}\PY{l+m+mi}{5}\PY{p}{)}
             \PY{k}{return} \PY{p}{[}\PY{n}{dist}\PY{p}{]}
\end{Verbatim}


    Note that we normalize the distributions because rounding error is
introduced when we calculate the probabilities in the distribution and
the sum probability of all possible events given a particular set of
causes must sum exactly to 1. Normalization of our pdfs is especially
important because we define the distribution directly rather than via
log values, which would be more accurate, but either way the pgmpy
library forces us to validate our distribution. Additionally, if in the
process of calculating a particular distribution we end up overflowing
our floats, we fall back to using integer math, which has the side
effect of introducing more error. Normalizing our distribution at the
end solves all these problems simultaneously. We then place the range of
our values into a dictionary and define conveneint functions for
generating prior and conditional probabilities for our model.

    \begin{Verbatim}[commandchars=\\\{\}]
{\color{incolor}In [{\color{incolor}26}]:} \PY{c+c1}{\PYZsh{} makes sure the values in the distribution all sum exactly to 1}
         \PY{k}{def} \PY{n+nf}{fix\PYZus{}dist\PYZus{}round\PYZus{}error}\PY{p}{(}\PY{n}{dist}\PY{p}{)}\PY{p}{:}
             \PY{n}{dist\PYZus{}sum} \PY{o}{=} \PY{n}{np}\PY{o}{.}\PY{n}{sum}\PY{p}{(}\PY{n}{dist}\PY{p}{)}
             \PY{k}{if} \PY{n}{dist\PYZus{}sum} \PY{o}{\PYZlt{}} \PY{l+m+mf}{1.0}\PY{p}{:}
                 \PY{n}{was\PYZus{}fixed} \PY{o}{=} \PY{k+kc}{True}
                 \PY{n}{dist}\PY{p}{[}\PY{n+nb}{len}\PY{p}{(}\PY{n}{dist}\PY{p}{)}\PY{o}{\PYZhy{}}\PY{l+m+mi}{1}\PY{p}{]} \PY{o}{=} \PY{l+m+mf}{1.0}\PY{o}{\PYZhy{}}\PY{n}{dist\PYZus{}sum} \PY{o}{+} \PY{n}{dist}\PY{p}{[}\PY{n+nb}{len}\PY{p}{(}\PY{n}{dist}\PY{p}{)}\PY{o}{\PYZhy{}}\PY{l+m+mi}{1}\PY{p}{]}
             \PY{n}{last\PYZus{}nonzero} \PY{o}{=} \PY{n+nb}{len}\PY{p}{(}\PY{n}{dist}\PY{p}{)}\PY{o}{\PYZhy{}}\PY{l+m+mi}{1}
             \PY{k}{while} \PY{n}{dist\PYZus{}sum} \PY{o}{\PYZgt{}} \PY{l+m+mf}{1.0}\PY{p}{:}
                 \PY{n}{was\PYZus{}fixed} \PY{o}{=} \PY{k+kc}{True}
                 \PY{c+c1}{\PYZsh{} print(dist\PYZus{}sum)}
                 \PY{n}{dist}\PY{p}{[}\PY{n}{last\PYZus{}nonzero}\PY{p}{]} \PY{o}{=} \PY{n+nb}{max}\PY{p}{(}\PY{l+m+mi}{0}\PY{p}{,} \PY{n}{dist}\PY{p}{[}\PY{n}{last\PYZus{}nonzero}\PY{p}{]}\PY{o}{\PYZhy{}}\PY{p}{(}\PY{n}{dist\PYZus{}sum}\PY{o}{\PYZhy{}}\PY{l+m+mf}{1.0}\PY{p}{)}\PY{p}{)}
                 \PY{n}{last\PYZus{}nonzero} \PY{o}{\PYZhy{}}\PY{o}{=} \PY{l+m+mi}{1}
                 \PY{n}{dist\PYZus{}sum} \PY{o}{=} \PY{n}{np}\PY{o}{.}\PY{n}{sum}\PY{p}{(}\PY{n}{dist}\PY{p}{)}
             \PY{k}{if} \PY{n}{np}\PY{o}{.}\PY{n}{sum}\PY{p}{(}\PY{n}{dist}\PY{p}{)} \PY{o}{!=} \PY{l+m+mf}{1.0}\PY{p}{:}
                 \PY{k}{return} \PY{n}{fix\PYZus{}dist\PYZus{}round\PYZus{}error}\PY{p}{(}\PY{n}{dist}\PY{p}{)}
             \PY{n+nb}{print}\PY{p}{(}\PY{n}{np}\PY{o}{.}\PY{n}{sum}\PY{p}{(}\PY{n}{dist}\PY{p}{)}\PY{p}{)}
             \PY{k}{return} \PY{n}{dist}
         
         \PY{c+c1}{\PYZsh{} define the range (ie number of possible values) of all variables}
         \PY{c+c1}{\PYZsh{} the range of all binary variables is 2 values, true or false; we redundantly define them here for clarity}
         \PY{n}{ranges} \PY{o}{=} \PY{p}{\PYZob{}}\PY{l+s+s1}{\PYZsq{}}\PY{l+s+s1}{V}\PY{l+s+s1}{\PYZsq{}}\PY{p}{:} \PY{l+m+mi}{2}\PY{p}{,} \PY{l+s+s1}{\PYZsq{}}\PY{l+s+s1}{S}\PY{l+s+s1}{\PYZsq{}}\PY{p}{:} \PY{l+m+mi}{2}\PY{p}{,} \PY{l+s+s1}{\PYZsq{}}\PY{l+s+s1}{R}\PY{l+s+s1}{\PYZsq{}}\PY{p}{:} \PY{l+m+mi}{2}\PY{p}{,} \PY{l+s+s1}{\PYZsq{}}\PY{l+s+s1}{C}\PY{l+s+s1}{\PYZsq{}}\PY{p}{:} \PY{l+m+mi}{2}\PY{p}{,} \PY{l+s+s1}{\PYZsq{}}\PY{l+s+s1}{U}\PY{l+s+s1}{\PYZsq{}}\PY{p}{:} \PY{l+m+mi}{2}\PY{p}{,} \PY{l+s+s1}{\PYZsq{}}\PY{l+s+s1}{D}\PY{l+s+s1}{\PYZsq{}}\PY{p}{:} \PY{l+m+mi}{3}\PY{p}{,} \PY{l+s+s1}{\PYZsq{}}\PY{l+s+s1}{A}\PY{l+s+s1}{\PYZsq{}}\PY{p}{:} \PY{l+m+mi}{10}\PY{p}{,} \PY{l+s+s1}{\PYZsq{}}\PY{l+s+s1}{E}\PY{l+s+s1}{\PYZsq{}}\PY{p}{:} \PY{l+m+mi}{5}\PY{p}{,} \PY{l+s+s1}{\PYZsq{}}\PY{l+s+s1}{L}\PY{l+s+s1}{\PYZsq{}}\PY{p}{:} \PY{l+m+mi}{10}\PY{p}{\PYZcb{}}
         
         \PY{c+c1}{\PYZsh{} prior probability}
         \PY{c+c1}{\PYZsh{} we normalize because rounding error is introduced when generating the distribution}
         \PY{k}{def} \PY{n+nf}{ppd}\PY{p}{(}\PY{n}{prior}\PY{p}{,} \PY{n}{distribution}\PY{p}{)}\PY{p}{:}
             \PY{n}{dist} \PY{o}{=} \PY{n}{tcpd}\PY{p}{(}\PY{n}{variable}\PY{o}{=}\PY{n}{prior}\PY{p}{,}
                         \PY{n}{variable\PYZus{}card}\PY{o}{=}\PY{n}{ranges}\PY{p}{[}\PY{n}{prior}\PY{p}{]}\PY{p}{,}
                         \PY{n}{values}\PY{o}{=}\PY{n}{distribution}\PY{p}{(}\PY{n}{ranges}\PY{p}{[}\PY{n}{prior}\PY{p}{]}\PY{p}{)}\PY{p}{)}
             \PY{n}{dist}\PY{o}{.}\PY{n}{normalize}\PY{p}{(}\PY{p}{)}
             \PY{k}{return} \PY{n}{dist}
         
         \PY{c+c1}{\PYZsh{} conditional probability}
         \PY{c+c1}{\PYZsh{} we normalize because rounding error is introduced when generating the distribution}
         \PY{k}{def} \PY{n+nf}{cpd}\PY{p}{(}\PY{n}{distribution}\PY{p}{,} \PY{o}{*}\PY{n}{variables}\PY{p}{)}\PY{p}{:}
             \PY{n}{dist} \PY{o}{=} \PY{n}{tcpd}\PY{p}{(}\PY{n}{variable}\PY{o}{=}\PY{n}{variables}\PY{p}{[}\PY{l+m+mi}{0}\PY{p}{]}\PY{p}{,} \PY{n}{variable\PYZus{}card}\PY{o}{=}\PY{n}{ranges}\PY{p}{[}\PY{n}{variables}\PY{p}{[}\PY{l+m+mi}{0}\PY{p}{]}\PY{p}{]}\PY{p}{,}
                        \PY{n}{evidence}\PY{o}{=}\PY{n}{variables}\PY{p}{[}\PY{l+m+mi}{1}\PY{p}{:}\PY{p}{]}\PY{p}{,} \PY{n}{evidence\PYZus{}card}\PY{o}{=}\PY{p}{[}\PY{n}{ranges}\PY{p}{[}\PY{n}{i}\PY{p}{]} \PY{k}{for} \PY{n}{i} \PY{o+ow}{in} \PY{n}{variables}\PY{p}{[}\PY{l+m+mi}{1}\PY{p}{:}\PY{p}{]}\PY{p}{]}\PY{p}{,}
                        \PY{n}{values}\PY{o}{=}\PY{n}{distribution}\PY{p}{(}\PY{n}{np}\PY{o}{.}\PY{n}{prod}\PY{p}{(}\PY{p}{[}\PY{n}{ranges}\PY{p}{[}\PY{n}{i}\PY{p}{]} \PY{k}{for} \PY{n}{i} \PY{o+ow}{in} \PY{n}{variables}\PY{p}{]}\PY{p}{)}\PY{p}{)}\PY{p}{)}
             \PY{n}{dist}\PY{o}{.}\PY{n}{normalize}\PY{p}{(}\PY{p}{)}
             \PY{k}{return} \PY{n}{dist}
\end{Verbatim}


    Defining our model is then just a matter of defining the causal
relationships between variables and calling the above helpers to
generate their prior and conditional probability distributions:

    \begin{Verbatim}[commandchars=\\\{\}]
{\color{incolor}In [{\color{incolor}27}]:} \PY{c+c1}{\PYZsh{} define model with connections between variables}
         \PY{n}{model} \PY{o}{=} \PY{n}{bysmodel}\PY{p}{(}\PY{p}{[}\PY{p}{(}\PY{l+s+s1}{\PYZsq{}}\PY{l+s+s1}{D}\PY{l+s+s1}{\PYZsq{}}\PY{p}{,} \PY{l+s+s1}{\PYZsq{}}\PY{l+s+s1}{V}\PY{l+s+s1}{\PYZsq{}}\PY{p}{)}\PY{p}{,} \PY{p}{(}\PY{l+s+s1}{\PYZsq{}}\PY{l+s+s1}{A}\PY{l+s+s1}{\PYZsq{}}\PY{p}{,} \PY{l+s+s1}{\PYZsq{}}\PY{l+s+s1}{V}\PY{l+s+s1}{\PYZsq{}}\PY{p}{)}\PY{p}{,} \PY{p}{(}\PY{l+s+s1}{\PYZsq{}}\PY{l+s+s1}{E}\PY{l+s+s1}{\PYZsq{}}\PY{p}{,} \PY{l+s+s1}{\PYZsq{}}\PY{l+s+s1}{V}\PY{l+s+s1}{\PYZsq{}}\PY{p}{)}\PY{p}{,} \PY{p}{(}\PY{l+s+s1}{\PYZsq{}}\PY{l+s+s1}{L}\PY{l+s+s1}{\PYZsq{}}\PY{p}{,} \PY{l+s+s1}{\PYZsq{}}\PY{l+s+s1}{D}\PY{l+s+s1}{\PYZsq{}}\PY{p}{)}\PY{p}{,} 
                           \PY{p}{(}\PY{l+s+s1}{\PYZsq{}}\PY{l+s+s1}{L}\PY{l+s+s1}{\PYZsq{}}\PY{p}{,} \PY{l+s+s1}{\PYZsq{}}\PY{l+s+s1}{A}\PY{l+s+s1}{\PYZsq{}}\PY{p}{)}\PY{p}{,} \PY{p}{(}\PY{l+s+s1}{\PYZsq{}}\PY{l+s+s1}{U}\PY{l+s+s1}{\PYZsq{}}\PY{p}{,} \PY{l+s+s1}{\PYZsq{}}\PY{l+s+s1}{D}\PY{l+s+s1}{\PYZsq{}}\PY{p}{)}\PY{p}{,} \PY{p}{(}\PY{l+s+s1}{\PYZsq{}}\PY{l+s+s1}{U}\PY{l+s+s1}{\PYZsq{}}\PY{p}{,} \PY{l+s+s1}{\PYZsq{}}\PY{l+s+s1}{L}\PY{l+s+s1}{\PYZsq{}}\PY{p}{)}\PY{p}{,} \PY{p}{(}\PY{l+s+s1}{\PYZsq{}}\PY{l+s+s1}{U}\PY{l+s+s1}{\PYZsq{}}\PY{p}{,} \PY{l+s+s1}{\PYZsq{}}\PY{l+s+s1}{A}\PY{l+s+s1}{\PYZsq{}}\PY{p}{)}\PY{p}{,}
                           \PY{p}{(}\PY{l+s+s1}{\PYZsq{}}\PY{l+s+s1}{S}\PY{l+s+s1}{\PYZsq{}}\PY{p}{,} \PY{l+s+s1}{\PYZsq{}}\PY{l+s+s1}{U}\PY{l+s+s1}{\PYZsq{}}\PY{p}{)}\PY{p}{,} \PY{p}{(}\PY{l+s+s1}{\PYZsq{}}\PY{l+s+s1}{R}\PY{l+s+s1}{\PYZsq{}}\PY{p}{,} \PY{l+s+s1}{\PYZsq{}}\PY{l+s+s1}{U}\PY{l+s+s1}{\PYZsq{}}\PY{p}{)}\PY{p}{,} \PY{p}{(}\PY{l+s+s1}{\PYZsq{}}\PY{l+s+s1}{C}\PY{l+s+s1}{\PYZsq{}}\PY{p}{,} \PY{l+s+s1}{\PYZsq{}}\PY{l+s+s1}{U}\PY{l+s+s1}{\PYZsq{}}\PY{p}{)}\PY{p}{]}\PY{p}{)}
\end{Verbatim}


    \begin{Verbatim}[commandchars=\\\{\}]
{\color{incolor}In [{\color{incolor}28}]:} \PY{c+c1}{\PYZsh{} define our prior and conditional pdfs}
         \PY{n}{priorS} \PY{o}{=} \PY{n}{ppd}\PY{p}{(}\PY{l+s+s1}{\PYZsq{}}\PY{l+s+s1}{S}\PY{l+s+s1}{\PYZsq{}}\PY{p}{,} \PY{n}{decreasing\PYZus{}distribution}\PY{p}{)}
         \PY{n}{priorR} \PY{o}{=} \PY{n}{ppd}\PY{p}{(}\PY{l+s+s1}{\PYZsq{}}\PY{l+s+s1}{R}\PY{l+s+s1}{\PYZsq{}}\PY{p}{,} \PY{n}{decreasing\PYZus{}distribution}\PY{p}{)}
         \PY{n}{priorC} \PY{o}{=} \PY{n}{ppd}\PY{p}{(}\PY{l+s+s1}{\PYZsq{}}\PY{l+s+s1}{C}\PY{l+s+s1}{\PYZsq{}}\PY{p}{,} \PY{n}{decreasing\PYZus{}distribution}\PY{p}{)}
         \PY{n}{priorE} \PY{o}{=} \PY{n}{ppd}\PY{p}{(}\PY{l+s+s1}{\PYZsq{}}\PY{l+s+s1}{E}\PY{l+s+s1}{\PYZsq{}}\PY{p}{,} \PY{n}{regular\PYZus{}distribution}\PY{p}{)}
         \PY{n}{cpdV} \PY{o}{=} \PY{n}{cpd}\PY{p}{(}\PY{n}{decreasing\PYZus{}distribution}\PY{p}{,} \PY{l+s+s1}{\PYZsq{}}\PY{l+s+s1}{V}\PY{l+s+s1}{\PYZsq{}}\PY{p}{,} \PY{l+s+s1}{\PYZsq{}}\PY{l+s+s1}{D}\PY{l+s+s1}{\PYZsq{}}\PY{p}{,} \PY{l+s+s1}{\PYZsq{}}\PY{l+s+s1}{A}\PY{l+s+s1}{\PYZsq{}}\PY{p}{,} \PY{l+s+s1}{\PYZsq{}}\PY{l+s+s1}{E}\PY{l+s+s1}{\PYZsq{}}\PY{p}{)}
         \PY{n}{cpdA} \PY{o}{=} \PY{n}{cpd}\PY{p}{(}\PY{n}{decreasing\PYZus{}distribution}\PY{p}{,} \PY{l+s+s1}{\PYZsq{}}\PY{l+s+s1}{A}\PY{l+s+s1}{\PYZsq{}}\PY{p}{,} \PY{l+s+s1}{\PYZsq{}}\PY{l+s+s1}{L}\PY{l+s+s1}{\PYZsq{}}\PY{p}{,} \PY{l+s+s1}{\PYZsq{}}\PY{l+s+s1}{U}\PY{l+s+s1}{\PYZsq{}}\PY{p}{)}
         \PY{n}{cpdD} \PY{o}{=} \PY{n}{cpd}\PY{p}{(}\PY{n}{regular\PYZus{}distribution}\PY{p}{,} \PY{l+s+s1}{\PYZsq{}}\PY{l+s+s1}{D}\PY{l+s+s1}{\PYZsq{}}\PY{p}{,} \PY{l+s+s1}{\PYZsq{}}\PY{l+s+s1}{L}\PY{l+s+s1}{\PYZsq{}}\PY{p}{,} \PY{l+s+s1}{\PYZsq{}}\PY{l+s+s1}{U}\PY{l+s+s1}{\PYZsq{}}\PY{p}{)}
         \PY{n}{cpdL} \PY{o}{=} \PY{n}{cpd}\PY{p}{(}\PY{n}{regular\PYZus{}distribution}\PY{p}{,} \PY{l+s+s1}{\PYZsq{}}\PY{l+s+s1}{L}\PY{l+s+s1}{\PYZsq{}}\PY{p}{,} \PY{l+s+s1}{\PYZsq{}}\PY{l+s+s1}{U}\PY{l+s+s1}{\PYZsq{}}\PY{p}{)}
         \PY{n}{cpdU} \PY{o}{=} \PY{n}{cpd}\PY{p}{(}\PY{n}{decreasing\PYZus{}distribution}\PY{p}{,} \PY{l+s+s1}{\PYZsq{}}\PY{l+s+s1}{U}\PY{l+s+s1}{\PYZsq{}}\PY{p}{,} \PY{l+s+s1}{\PYZsq{}}\PY{l+s+s1}{S}\PY{l+s+s1}{\PYZsq{}}\PY{p}{,} \PY{l+s+s1}{\PYZsq{}}\PY{l+s+s1}{R}\PY{l+s+s1}{\PYZsq{}}\PY{p}{,} \PY{l+s+s1}{\PYZsq{}}\PY{l+s+s1}{C}\PY{l+s+s1}{\PYZsq{}}\PY{p}{)}
         
         \PY{c+c1}{\PYZsh{} add distributions to the model}
         \PY{n}{model}\PY{o}{.}\PY{n}{add\PYZus{}cpds}\PY{p}{(}\PY{n}{priorS}\PY{p}{,} \PY{n}{priorR}\PY{p}{,} \PY{n}{priorC}\PY{p}{,} \PY{n}{priorE}\PY{p}{,} \PY{n}{cpdV}\PY{p}{,} \PY{n}{cpdA}\PY{p}{,} \PY{n}{cpdD}\PY{p}{,} \PY{n}{cpdL}\PY{p}{,} \PY{n}{cpdU}\PY{p}{)}
         \PY{c+c1}{\PYZsh{} check consistency}
         \PY{n}{model}\PY{o}{.}\PY{n}{check\PYZus{}model}\PY{p}{(}\PY{p}{)}
\end{Verbatim}


\begin{Verbatim}[commandchars=\\\{\}]
{\color{outcolor}Out[{\color{outcolor}28}]:} True
\end{Verbatim}
            
    \hypertarget{solving-an-example-problem-comparison-of-methods}{%
\subsection{Solving an Example Problem: Comparison of
Methods}\label{solving-an-example-problem-comparison-of-methods}}

    Given this network, let us compare different methods of computing a
particular conditional probability. Let's suppose we want to figure out
the most probable network load (\(L\)) given that the sender and
receiver are both up (\(S,R\)), that 2 duplicate packets have been sent
thus far (\(D\)) and that packet transmission was successful (\(V\)). In
short, we want to derive the conditional probability distribution\\
\(P(L|S,R,D=2,V)\)

    \hypertarget{variable-elimination}{%
\subsubsection{Variable Elimination}\label{variable-elimination}}

    \begin{Verbatim}[commandchars=\\\{\}]
{\color{incolor}In [{\color{incolor}29}]:} \PY{k}{def} \PY{n+nf}{solve\PYZus{}and\PYZus{}display}\PY{p}{(}\PY{n}{solver}\PY{p}{,} \PY{n}{model}\PY{p}{)}\PY{p}{:}
             \PY{n}{distvals} \PY{o}{=} \PY{p}{(}\PY{n}{solver}\PY{o}{.}\PY{n}{query}\PY{p}{(}\PY{p}{[}\PY{l+s+s1}{\PYZsq{}}\PY{l+s+s1}{L}\PY{l+s+s1}{\PYZsq{}}\PY{p}{]}\PY{p}{,}
                               \PY{n}{evidence}\PY{o}{=}\PY{p}{\PYZob{}}\PY{l+s+s1}{\PYZsq{}}\PY{l+s+s1}{S}\PY{l+s+s1}{\PYZsq{}} \PY{p}{:} \PY{l+m+mi}{1}\PY{p}{,} \PY{l+s+s1}{\PYZsq{}}\PY{l+s+s1}{R}\PY{l+s+s1}{\PYZsq{}} \PY{p}{:} \PY{l+m+mi}{1}\PY{p}{,} \PY{l+s+s1}{\PYZsq{}}\PY{l+s+s1}{D}\PY{l+s+s1}{\PYZsq{}} \PY{p}{:} \PY{l+m+mi}{2}\PY{p}{,} \PY{l+s+s1}{\PYZsq{}}\PY{l+s+s1}{V}\PY{l+s+s1}{\PYZsq{}} \PY{p}{:} \PY{l+m+mi}{1}\PY{p}{\PYZcb{}}\PY{p}{)}
                                          \PY{p}{[}\PY{l+s+s1}{\PYZsq{}}\PY{l+s+s1}{L}\PY{l+s+s1}{\PYZsq{}}\PY{p}{]}\PY{o}{.}\PY{n}{values}\PY{p}{)}
             \PY{n}{plt}\PY{o}{.}\PY{n}{bar}\PY{p}{(}\PY{n+nb}{range}\PY{p}{(}\PY{l+m+mi}{0}\PY{p}{,}\PY{n}{ranges}\PY{p}{[}\PY{l+s+s1}{\PYZsq{}}\PY{l+s+s1}{L}\PY{l+s+s1}{\PYZsq{}}\PY{p}{]}\PY{p}{)}\PY{p}{,} \PY{n}{distvals}\PY{p}{)}
             \PY{n}{plt}\PY{o}{.}\PY{n}{grid}\PY{p}{(}\PY{p}{)}
             \PY{n}{plt}\PY{o}{.}\PY{n}{show}\PY{p}{(}\PY{p}{)}
\end{Verbatim}


    \begin{Verbatim}[commandchars=\\\{\}]
{\color{incolor}In [{\color{incolor}33}]:} \PY{k+kn}{from} \PY{n+nn}{pgmpy}\PY{n+nn}{.}\PY{n+nn}{inference} \PY{k}{import} \PY{n}{VariableElimination}
         
         \PY{n}{VESolver} \PY{o}{=} \PY{n}{VariableElimination}\PY{p}{(}\PY{n}{model}\PY{p}{)}
         \PY{n}{solve\PYZus{}and\PYZus{}display}\PY{p}{(}\PY{n}{VESolver}\PY{p}{,} \PY{n}{model}\PY{p}{)}
\end{Verbatim}


    \begin{center}
    \adjustimage{max size={0.9\linewidth}{0.9\paperheight}}{output_23_0.png}
    \end{center}
    { \hspace*{\fill} \\}
    
    This is what we would expect, as the sender and receiver being up and
successful packet transmission is likely to be at the same time as a
lighter network load, which here has a pdf with a mean around 2 as
opposed to half its maximum value, 5. The Variable Elimination solver is
able to derive the resulting distribution in negligible time, making it
a suitable method of derivation.

    \hypertarget{belief-propagation}{%
\subsubsection{Belief Propagation}\label{belief-propagation}}

    \begin{Verbatim}[commandchars=\\\{\}]
{\color{incolor}In [{\color{incolor}34}]:} \PY{k+kn}{from} \PY{n+nn}{pgmpy}\PY{n+nn}{.}\PY{n+nn}{inference} \PY{k}{import} \PY{n}{BeliefPropagation}
         
         \PY{n}{BPSolver} \PY{o}{=} \PY{n}{BeliefPropagation}\PY{p}{(}\PY{n}{model}\PY{p}{)}
         \PY{n}{BPSolver}\PY{o}{.}\PY{n}{calibrate}\PY{p}{(}\PY{p}{)}
         \PY{n}{solve\PYZus{}and\PYZus{}display}\PY{p}{(}\PY{n}{BPSolver}\PY{p}{,} \PY{n}{model}\PY{p}{)}
\end{Verbatim}


    \begin{center}
    \adjustimage{max size={0.9\linewidth}{0.9\paperheight}}{output_26_0.png}
    \end{center}
    { \hspace*{\fill} \\}
    
    Similarly, running belief propagation on the model for the same query is
nearly instantaneous and thus an effective method of deriving
probabilities given this particular case.

    \hypertarget{bayesian-model-sampling}{%
\subsubsection{Bayesian Model Sampling}\label{bayesian-model-sampling}}

    \begin{Verbatim}[commandchars=\\\{\}]
{\color{incolor}In [{\color{incolor} }]:} \PY{k+kn}{from} \PY{n+nn}{pgmpy}\PY{n+nn}{.}\PY{n+nn}{factors}\PY{n+nn}{.}\PY{n+nn}{discrete} \PY{k}{import} \PY{n}{State}
        \PY{k+kn}{from} \PY{n+nn}{pgmpy}\PY{n+nn}{.}\PY{n+nn}{sampling} \PY{k}{import} \PY{n}{BayesianModelSampling}
        \PY{k+kn}{from} \PY{n+nn}{pandas}\PY{n+nn}{.}\PY{n+nn}{core}\PY{n+nn}{.}\PY{n+nn}{frame} \PY{k}{import} \PY{n}{DataFrame}
        
        
        \PY{k}{def} \PY{n+nf}{condProb}\PY{p}{(}\PY{n}{trace}\PY{p}{,} \PY{n}{event}\PY{p}{,} \PY{n}{cond}\PY{p}{)}\PY{p}{:}
            \PY{k}{if} \PY{n+nb}{type}\PY{p}{(}\PY{n}{trace}\PY{p}{)} \PY{o+ow}{is} \PY{n}{DataFrame}\PY{p}{:}
                \PY{n}{trace} \PY{o}{=} \PY{n}{trace}\PY{o}{.}\PY{n}{transpose}\PY{p}{(}\PY{p}{)}\PY{o}{.}\PY{n}{to\PYZus{}dict}\PY{p}{(}\PY{p}{)}\PY{o}{.}\PY{n}{values}\PY{p}{(}\PY{p}{)}
            \PY{c+c1}{\PYZsh{} find all samples satisfy conditions}
            \PY{k}{for} \PY{n}{k}\PY{p}{,} \PY{n}{v} \PY{o+ow}{in} \PY{n}{cond}\PY{o}{.}\PY{n}{items}\PY{p}{(}\PY{p}{)}\PY{p}{:}
                \PY{n}{trace} \PY{o}{=} \PY{p}{[}\PY{n}{smp} \PY{k}{for} \PY{n}{smp} \PY{o+ow}{in} \PY{n}{trace} \PY{k}{if} \PY{n}{smp}\PY{p}{[}\PY{n}{k}\PY{p}{]} \PY{o}{==} \PY{n}{v}\PY{p}{]}
            \PY{c+c1}{\PYZsh{} record quantity of all samples fulfill condition}
            \PY{n}{nCondSample} \PY{o}{=} \PY{n+nb}{len}\PY{p}{(}\PY{n}{trace}\PY{p}{)}
            \PY{c+c1}{\PYZsh{} find all samples satisfy event}
            \PY{k}{for} \PY{n}{k}\PY{p}{,} \PY{n}{v} \PY{o+ow}{in} \PY{n}{event}\PY{o}{.}\PY{n}{items}\PY{p}{(}\PY{p}{)}\PY{p}{:}
                \PY{n}{trace} \PY{o}{=} \PY{p}{[}\PY{n}{smp} \PY{k}{for} \PY{n}{smp} \PY{o+ow}{in} \PY{n}{trace} \PY{k}{if} \PY{n}{smp}\PY{p}{[}\PY{n}{k}\PY{p}{]} \PY{o}{==} \PY{n}{v}\PY{p}{]}
            \PY{c+c1}{\PYZsh{} calculate conditional probability}
            \PY{k}{return} \PY{n+nb}{len}\PY{p}{(}\PY{n}{trace}\PY{p}{)} \PY{o}{/} \PY{n}{nCondSample}
        
        \PY{k}{class} \PY{n+nc}{MonteCarlo}\PY{p}{:}
            \PY{n}{model} \PY{o}{=} \PY{k+kc}{None}
            \PY{n}{nsamples} \PY{o}{=} \PY{l+m+mi}{100}\PY{p}{;}
            \PY{n}{solverComponent} \PY{o}{=} \PY{k+kc}{None}
            
            \PY{k}{def} \PY{n+nf}{\PYZus{}\PYZus{}init\PYZus{}\PYZus{}}\PY{p}{(}\PY{n+nb+bp}{self}\PY{p}{,} \PY{n}{bysmodel}\PY{p}{,} \PY{n}{nsamples}\PY{o}{=}\PY{l+m+mi}{100}\PY{p}{)}\PY{p}{:}
                \PY{n+nb+bp}{self}\PY{o}{.}\PY{n}{model} \PY{o}{=} \PY{n}{bysmodel}
                \PY{n+nb+bp}{self}\PY{o}{.}\PY{n}{nsamples} \PY{o}{=} \PY{n}{nsamples}
                \PY{n+nb+bp}{self}\PY{o}{.}\PY{n}{solverComponent} \PY{o}{=} \PY{n}{BayesianModelSampling}\PY{p}{(}\PY{n}{model}\PY{p}{)}
            
            \PY{c+c1}{\PYZsh{} NOTE: query here only works for 1 query variable, but that\PYZsq{}s fine for this example}
            \PY{k}{def} \PY{n+nf}{query}\PY{p}{(}\PY{n+nb+bp}{self}\PY{p}{,} \PY{n}{qvars}\PY{p}{,} \PY{n}{evidence}\PY{p}{)}\PY{p}{:}
                \PY{n}{qvar} \PY{o}{=} \PY{n}{qvars}\PY{p}{[}\PY{l+m+mi}{0}\PY{p}{]}
                \PY{n}{evars} \PY{o}{=} \PY{n}{evidence}
                \PY{n}{estates} \PY{o}{=} \PY{p}{[}\PY{n}{State}\PY{p}{(}\PY{n}{e}\PY{p}{,} \PY{n}{evars}\PY{p}{[}\PY{n}{e}\PY{p}{]}\PY{p}{)} \PY{k}{for} \PY{n}{e} \PY{o+ow}{in} \PY{n}{evars}\PY{o}{.}\PY{n}{keys}\PY{p}{(}\PY{p}{)}\PY{p}{]}
                \PY{c+c1}{\PYZsh{} NOTE: this will take a very very long time}
                \PY{n+nb}{print}\PY{p}{(}\PY{l+s+s2}{\PYZdq{}}\PY{l+s+s2}{about to rejection sample...}\PY{l+s+s2}{\PYZdq{}}\PY{p}{)}
                \PY{n}{sample} \PY{o}{=} \PY{n+nb+bp}{self}\PY{o}{.}\PY{n}{solverComponent}\PY{o}{.}\PY{n}{rejection\PYZus{}sample}\PY{p}{(}
                    \PY{n}{evidence}\PY{o}{=}\PY{n}{estates}\PY{p}{,}
                    \PY{n}{size}\PY{o}{=}\PY{n+nb+bp}{self}\PY{o}{.}\PY{n}{nsamples}\PY{p}{)}
                \PY{n+nb}{print}\PY{p}{(}\PY{l+s+s2}{\PYZdq{}}\PY{l+s+s2}{rejection sample complete}\PY{l+s+s2}{\PYZdq{}}\PY{p}{)}
                \PY{k}{return} \PY{p}{[}\PY{n}{condProb}\PY{p}{(}\PY{n}{sample}\PY{p}{,} \PY{p}{\PYZob{}}\PY{n}{qvar} \PY{p}{:} \PY{n}{val}\PY{p}{\PYZcb{}}\PY{p}{,} \PY{n}{evars}\PY{p}{)} \PY{k}{for} \PY{n}{val} \PY{o+ow}{in} \PY{n+nb}{range}\PY{p}{(}\PY{n}{ranges}\PY{p}{[}\PY{n}{qvar}\PY{p}{]}\PY{p}{)}\PY{p}{]}
            
        
        \PY{n}{MCSolver} \PY{o}{=} \PY{n}{MonteCarlo}\PY{p}{(}\PY{n}{model}\PY{p}{)}
        \PY{n}{solve\PYZus{}and\PYZus{}display}\PY{p}{(}\PY{n}{MCSolver}\PY{p}{,} \PY{n}{model}\PY{p}{)}
\end{Verbatim}


    \begin{Verbatim}[commandchars=\\\{\}]
about to rejection sample{\ldots}

    \end{Verbatim}

    Note that this method takes noticably longer than the other two when
calculating the conditional probability distribution, and so is probably
not an effective means of deriving probability distributions for the
case of this particular model.


    % Add a bibliography block to the postdoc
    
    
    
    \end{document}
